% thesisabstract.tex
% Describes abstract of document.

\abstract
\begin{changemargin}{+1cm}{+1cm}
\noindent
\textbf{Context}.
Full-system simulators provide benefits to developers in terms of a more sustainable development cycle; since development may begin prior to that of next-generation hardware being available.
However, there is a distinct lack of graphics virtualization in industry-grade virtual platforms, leading to performance issues that may obfuscate the benefits virtual platforms otherwize have over execution on actual hardware.
\newline
\textbf{Objectives}.
This dissertation concerns the implementation of graphics acceleration by the means of paravirtualization of \dvttermopenglestwopointo\ in the \dvttermsimics\ full-system simulator.
Furthermore, this study illustrates the benefits and drawbacks of paravirtualization as a method, in addition performance analysis and comparison with the \dvttermandroidemulator ; which likewize utilize paravirtualization to accelerate simulated graphics.
\newline
\textbf{Methods}.
For this study, a number of benchmarks have been devised to stress key points in the developed solution; comprising areas such as simulation- \dvttermtarget -to-\dvttermhost\ communication latency and bandwidth.
Additionally, the solution is benchmarked upon its capability to handle computational stress.
\newline
\textbf{Results}
For the purpose of this experiment, elapsed frametimes for respective benchmarks are collected and compared in-between that of four platforms; being the hardware accelerated \dvttermhost\ machine, the paravirtualized \dvttermandroidemulator , software rasterized \dvttermsimics\ and paravirtualized \dvttermsimics .
\newline
\textbf{Conclusions}.
This thesis establishes paravirtualization as a feasable method to achieve accelerated graphics in virtual platforms.
The study concludes graphics acceleration of up to $34$ times of that of its software rasterized counterparts, and predicts further improvement in future tests.
Furthermore, the study establishes \dvttermmagicinstruction s as the primary bottleneck of communication latency in the devised solution.

\par\vspace {0.5cm}
% 3-4 keywords, maximum 2 of these from the title, starts 1 line below the
% abstract.
\iftoggle{acmclassification}{
	\noindent
	\textbf{Classification:} E.1.1 [Software infrastructure]: Virtual machines; K.6.4 [Graphics systems and interfaces]: Graphics processors; N.1.0 [Companies]: Intel Corporation;\\
}
\noindent
\textbf{Keywords:} Paravirtualization; Simics; % 3-4 keywords, maximum 2 of these from the title, starts 1 line below the abstract.
\iftoggle{generalterms}{
	\\ \noindent
	\textbf{General Terms:} Deterministic Execution; Checkpointing; Reverse Execution;
}

\end{changemargin}

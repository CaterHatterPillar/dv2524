% thesisresults.tex
% Chapter Results.

% Results
\chapter{Results}
\label{cha:results}

% Test table
%% tabhost.tex
% Defines a table outlining the performance of the thesis benchmark whilst hardware accelerated on the host platform.

\begin{tabular}{|c|c|c|c|c|}
\hline
\multirow{2}{*}{Benchmark} & \multicolumn{4}{p{6cm}|}{\centering Elapsed time in Milliseconds} \\
\cline{2-5} & \multicolumn{1}{c|}{Min} & \multicolumn{1}{c|}{Max} & \multicolumn{1}{c|}{Std} & \multicolumn{1}{c|}{Avg} \\ \hline
Chess & \dvtcmdfirstline{hostchess84x84.dat.min}		& \dvtcmdfirstline{hostchess84x84.dat.max}		& \dvtcmdfirstline{hostchess84x84.dat.std}		& \dvtcmdfirstline{hostchess84x84.dat.avg} \\ \hline
Julia & \dvtcmdfirstline{hostjulia450.dat.min}			& \dvtcmdfirstline{hostjulia450.dat.max}		& \dvtcmdfirstline{hostjulia450.dat.std}		& \dvtcmdfirstline{hostjulia450.dat.avg} \\ \hline
Phong & \dvtcmdfirstline{hostphong2048x2048.dat.min}	& \dvtcmdfirstline{hostphong2048x2048.dat.max}	& \dvtcmdfirstline{hostphong2048x2048.dat.std}	& \dvtcmdfirstline{hostphong2048x2048.dat.avg} \\ \hline
\end{tabular}

% tabkeyvalsimics.tex
% Defines a table outlining the performance of the thesis benchmark whilst hardware accelerated on the host platform.

% This ought be made into an environment.

\begin{table}
\begin{center}

\newcommand{\chesskeyone}{60x60 tiles}
\newcommand{\chesskeytwo}{84x84 tiles}
\newcommand{\chesskeythree}{118x118 tiles}

\newcommand{\juliakeyone}{225 iterations}
\newcommand{\juliakeytwo}{450 iterations}
\newcommand{\juliakeythree}{900 iterations}

\newcommand{\phongkeyone}{1448x1448 texels}
\newcommand{\phongkeytwo}{2048x2048 texels}
\newcommand{\phongkeythree}{2896x2896 texels}

\begin{tabular}{|c|c|c|c|c|c|}
\hline
\multirow{2}{*}{Benchmark} & \multirow{2}{*}{Key} & \multicolumn{4}{p{6cm}|}{\centering Elapsed time in Milliseconds} \\
\cline{3-6} && \multicolumn{1}{c|}{Min} & \multicolumn{1}{c|}{Max} & \multicolumn{1}{c|}{Std} & \multicolumn{1}{c|}{Avg} \\ \hline
\multirow{3}{*}{Chess} & \chesskeyone & \dvtcmdfirstline{simicschess60x60.dat.min}		& \dvtcmdfirstline{simicschess60x60.dat.max}		& \dvtcmdfirstline{simicschess60x60.dat.std}		& \dvtcmdfirstline{simicschess60x60.dat.avg} \\ %\cline{2-6}
& \chesskeytwo & \dvtcmdfirstline{simicschess84x84.dat.min} & \dvtcmdfirstline{simicschess84x84.dat.max} & \dvtcmdfirstline{simicschess84x84.dat.std} & \dvtcmdfirstline{simicschess84x84.dat.avg} \\ %\cline{2-6}
& \chesskeythree & \dvtcmdfirstline{simicschess118x118.dat.min} & \dvtcmdfirstline{simicschess118x118.dat.max} & \dvtcmdfirstline{simicschess118x118.dat.std} & \dvtcmdfirstline{simicschess118x118.dat.avg} \\ \hline
\multirow{3}{*}{Julia} & \juliakeyone & \dvtcmdfirstline{simicsjulia225.dat.min}		& \dvtcmdfirstline{simicsjulia225.dat.max}		& \dvtcmdfirstline{simicsjulia225.dat.std} & \dvtcmdfirstline{simicsjulia225.dat.avg} \\ %\cline{2-6}
& \juliakeytwo & \dvtcmdfirstline{simicsjulia450.dat.min} & \dvtcmdfirstline{simicsjulia450.dat.max} & \dvtcmdfirstline{simicsjulia450.dat.std} & \dvtcmdfirstline{simicsjulia450.dat.avg} \\ %\cline{2-6}
& \juliakeythree & \dvtcmdfirstline{simicsjulia900.dat.min} & \dvtcmdfirstline{simicsjulia900.dat.max} & \dvtcmdfirstline{simicsjulia900.dat.std} & \dvtcmdfirstline{simicsjulia900.dat.avg} \\ \hline
\multirow{3}{*}{Phong} & \phongkeyone & \dvtcmdfirstline{simicsphong1448x1448.dat.min}		& \dvtcmdfirstline{simicsphong1448x1448.dat.max}		& \dvtcmdfirstline{simicsphong1448x1448.dat.std}		& \dvtcmdfirstline{simicsphong1448x1448.dat.avg} \\ %\cline{2-6}
& \phongkeytwo & \dvtcmdfirstline{simicsphong2048x2048.dat.min} & \dvtcmdfirstline{simicsphong2048x2048.dat.max} & \dvtcmdfirstline{simicsphong2048x2048.dat.std} & \dvtcmdfirstline{simicsphong2048x2048.dat.avg} \\ %\cline{2-6}
& \phongkeythree & \dvtcmdfirstline{simicsphong2896x2896.dat.min} & \dvtcmdfirstline{simicsphong2896x2896.dat.max} & \dvtcmdfirstline{simicsphong2896x2896.dat.std} & \dvtcmdfirstline{simicsphong2896x2896.dat.avg} \\ \hline
\end{tabular}

\label{tab:tabletest}
\caption{A test table}

\end{center}
\end{table}

% TODO:
% Below yadda yadda is visualized in histograms (see section Histograms for an elaboration on how these ought be interpreted).

% TODO:
% QEMU performance
% X11 Performance
% X11 complete performance under appendix or results?
% Magic instruction profiling
% Section for each demo? (analysis, rather?)

% hosthistograms
% fighistogramshost.tex

\begin{figure}
\begin{center}
	\input{gnuhistogramshost.tex}

	\caption{Histogram depicting benchmark elapsed frametimes in milliseconds and the density distribution of 1000 frames whilst hardware accelerated on the simulation host.}
	\label{fig:histogramshost}
\end{center}
\end{figure}

% qemuhistograms
% fighistogramsqemu.tex

\begin{figure}
\begin{center}
	\input{gnuhistogramsqemu.tex}

	\caption{Histogram depicting benchmark elapsed frametimes in milliseconds and the density distribution of 1000 frames whilst paravirtualized in QEMU.}
	\label{fig:histogramsqemu}
\end{center}
\end{figure}

% simicsparachesshistograms
% fighistogramssimicsparachess.tex

\begin{sidewaysfigure}
	\centering
	\input{gnuhistogramssimicsparachess.tex}

	\caption{Histograms depicting benchmark elapsed frametimes in milliseconds and the density distribution of 1000 frames for the Chess benchmark key figure variations whilst software rasterized- and paravirtualized in \dvttermsimics .}
	\label{fig:histogramssimicsparachess}
\end{sidewaysfigure}

% simicsparajuliahistograms
% fighistogramssimicsparajulia.tex

\begin{sidewaysfigure}
\begin{center}
	\input{gnuhistogramssimicsparajulia.tex}

    \label{fig:histogramssimicsparajulia}
    \caption{Left: Performance of the Julia benchmark whilst software rasterized in \dvttermsimics . Right: Performance of the Julia benchmark whilst paravirtualized in \dvttermsimics .}
\end{center}
\end{sidewaysfigure}

% simicsparaphonghistograms
% fighistogramssimicsparaphong.tex

\begin{sidewaysfigure}
\begin{center}
	\input{gnuhistogramssimicsparaphong.tex}

    \label{fig:histogramssimicsparaphong}
    \caption{Left: Performance of the Phong benchmark whilst software rasterized in \dvttermsimics . Right: Performance of the Phong benchmark whilst paravirtualized in \dvttermsimics .}
\end{center}
\end{sidewaysfigure}

% Histograms
\section{Histograms}
\label{sec:histograms}
The benchmarking results gathered for the purposes of this study, with the profiling methods having been described in section \dvtcmdrefname{sec:methodologyexperiment_platformprofiling}, are compiled into histograms; visualizing elapsed time in milliseconds to density in the figures presented in this section.
As such, the $Y$ axis showcases sample density; although the axis keys have been removed as they bear little relevance to the outcomes presented in this document.
The histograms each feature \dvtcmdnum{100} bins based off the \dvtcmdnum{1000} samples gathered per benchmark, as described in section \dvtcmdrefname{sec:methodologyexperiment_platformprofiling}, which are rounded into said bins.
For the purposes of good visualization methodology, values outside of the standard deviation\footnote{That is; values above that of the $mean + std$ and values under that of $mean - std$.} are not featured in the figures presented in this section.
In order to accomodate for the, however few, samples outside of said limits the figures are all complemented with key ratio tables (see \todo{Refer to tables} ).
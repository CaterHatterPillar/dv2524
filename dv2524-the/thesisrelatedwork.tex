% thesisrelatedwork.tex
% Chapter Related Work.

% TODO:
% Add Lagar-Cavilla as a section

\chapter{Related Work}
\label{cha:relatedwork}
System simulators are abundant and exist in corporate\dvtcmdcitebib{magazines:bohrer:2004}, academic\dvtcmdcitebib{journals:rosenblum:1995}, and open-source variations\dvtcmdciteref{magazines:bartholomew:2006}.
Such platforms, like \dvttermsimics , have been used for a variety of purposes including, but not limited to, thermal control strategies in multicores\dvtcmdcitebib{inproceedings:bartolini:2010}, networking timing analysis\dvtcmdcitebib{journals:ortiz:2009}, web server performance evaluation\dvtcmdcitebib{journals:villa:2005}, and to simulate costly hardware financially unfeasible to researchers\dvtcmdcitebib{journals:alameldeen:2003}.
Furthermore, such simulators are may be used to port \dvttermos s to new processors\dvtcmdciteref{technicaldocs:netbsd:2014}.\\

% Related work: Graphics acceleration
\noindent
For the purposes of graphics acceleration, there are number of stretegies, methodologies and procedures (several of these are expanded upon in chapter \ref{cha:methodologysolution}).
Based off these core strategies, such as device modeling, various passthrough technologies, and paravirtualization, there have been numerous attempts at effective \dvttermgpu\ virtualization; many of which require modification of both \dvttermtarget - and \dvttermhost\ systems; such as the development of specialized passthrough drivers\dvtcmdcitebib{inproceedings:lagarcavilla:2007}.
One such instace is presented by Hansen in his work on the Blink display system\dvtcmdcitebib{inproceedings:hansen:2007}.

Neither are the concepts of advanced simulatory features new to \dvttermgpu\ virtualization, as there have are multiple attempts to implement the \dvttermcheckpointrestart -model in a \dvttermgpu\ context, such as the work done by Guo et al. on the CUDA framework\dvtcmdcitebib{inproceedings:guo:2013}.
Another solution which also supports the \dvttermcheckpointrestart -scheme is the VMGL-solution, as presented by Lagar-Cavilla et al., which by the means of paravirtualization accelerated the \dvttermopengl\ $1.5$\dvtcmdcitebib{inproceedings:lagarcavilla:2007}.
The groundwork produced by Lagar-Cavilla et al. showcased the potential for paravirtualized graphics, as the VMGL framework, for a certain set of benchmarks, attained improvements of roughly two order of magnitude - in relation to software rasterization.

Other related works include modeling \dvttermgpu\ devices in the \dvttermqemu\ full-system simulator with software \dvttermopengles\ rasterization support, as presented by Shen et al.\dvtcmdcitebib{inproceedings:shen:2010}.\\

\noindent
Whatever the method, graphics virtualization is no longer limited to the academic community; but also in industry, as big virtualization players have begun to incorporate graphics accelerations support in their products - such as \dvttermvmware \dvtcmdciteref{technicaldocs:vmware:2014}.

% Stuff to look up:
% * Any more available info on Virgil3D?
% * Previous work on GPU Reverse Execution.
% * Previous work on Determinism in GPU technologies.
% * Previous work on GPU simulation, in general.

% Reference Solution
\section{QEMU}
\label{sec:relatedwork_qemu}
\index{QEMU}
\index{Reference Solution}
\index{Reference Implementation}
\dvttermqemu \footnote{'Quick~Emulator'.} is an open-source virtual platform described as a full system emulator\dvtcmdcitebib[p.~1]{inproceedings:bellard:2005} (see \dvtcmdciteref[p.~69]{magazines:bartholomew:2006} for an overview of \dvttermqemu ) and a high-speed functional simulator\dvtcmdcitebib[p.~1]{inproceedings:shen:2010}.
Its virtual platform supports simulation of several common system architectures and hardware devices and can, like \dvttermsimics , save and restore the state of a simulation\dvtcmdcitebib[p.~1]{inproceedings:bellard:2005}.

As such, \dvttermqemu\ may, like \dvttermsimics , run unmodified \dvttermtarget\ hardware - such as \dvttermos s, drivers, and other software.


%\dvttermqemu\ is an extensively used virtual platform, and is the simulator used in the \dvttermreferencesolution .

\paragraph{Android Emulator}
\label{par:relatedwork_qemu_androidemulator}
\index{Android Emulator}

The \dvttermandroidemulator\ is described as a virtual mobile device emulator\dvtcmdciteref{web:google:2013:usingtheemulator}. Included in the \dvttermandroidsdk , it supports virtualization of an assortment of mobile hardware permutations.
In the presence of thet \dvttermandroid\ \dvtcmdnum{4.0.3} release, the \dvttermandroidsdk\ was updated to make use of hardware-assisted \dvttermxeightysix\ virtualization; significantly increasing performance of \dvttermcpu -bound workloads\dvtcmdciteref{web:ducrohet:2012:afasteremulator}.
In addition to this, \dvttermgoogle\ implemented \dvttermopengles\ \dvtcmdnum{1.1} \& \dvtcmdnum{2.0} hardware acceleration; offering a substantial performance boost to developers utilizing the \dvttermopengles\ framework\dvtcmdciteref{web:ducrohet:2012:afasteremulator}.
\dvttermgoogle 's solution, the design documents of which are attached in \dvtcmdcitefur{dissertation:nilsson:2014} (see \dvtcmdcitefur{technicaldocs:google:2014}), consists of a paravirtualized implementation which circumvents the simulated system by forwarding its \dvttermopengles\ invocations to the \dvttermhost\ system directly via the simulator program.

As of \dvttermandroid\ \dvtcmdnum{4.4}, the \dvttermandroidemulator\ uses \dvttermqemu\ to simulate \dvttermarm\ and \dvttermxeightysix\ devices aiding those wishing to develop software for mobile units.
The \dvttermandroidemulator\ implementation and solution are referred to as the \dvttermreferenceimplementation\ and the \dvttermreferencesolution , respectively, throughout this document.

% http://lists.gnu.org/archive/html/qemu-devel/2013-07/msg02918.html

% Unsupported claims:
%Albeit supporting \termgpu\ virtualization, at version $1.7$ \termqemu\ still, in and of its own, lack support for hardware accelerated rasterization\cite{NEEDED}.
%However, as early as 2006 there were experimental attempts at accelerating \termopengl\ in \termqemu \cite{SEE}.
%The \termqemu\ platform has since been the focal point of several attempts at achieving such acceleration, possibly due to its open-source licensing.
%Such an attempt is the \termvirgilthreed\ virtual \termgpu-project\cite{SEE}, that strives to create an abstract \termtarget\ \termgpu\ which may utilize the \termhost\ device.

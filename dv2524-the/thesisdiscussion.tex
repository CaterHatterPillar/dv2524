% thesisdiscussion.tex

\chapter{Discussion}
\label{cha:discussion}
For the purposes of productification of a graphics acceleration solution, as presented in this document, necessary improvements include endianness\footnote{Which is, at the time of writing, assumed to be of little endian order.} consideration in the communication in-between \dvttermtarget - and \dvttermhost\ libraries, in addition to floating point format\footnote{Which is, at the time of writing, assumed to be that of \dvttermieeefp .}.
If paravirtualization were to be used for accelerating graphics in the \dvttermsimics\ full-system simulator, one would have to support such \dvttermtarget - and \dvttermhost\ architectures as it is not all uncommon that these platforms differ.
After all, clients often wish to simulate other platforms than those they currently possess.

Another cross-platform issue, that comes to mind when interfering with the \dvttermtarget\ physical memory from outside simulation, is that is \glslink{dvtglossmemorypagelocking}{memory page locking}.
Such functionalities are sometimes limited in operating systems; albeit entirely controlled by the user in \dvttermlinux\ derived systems.
\glslink{dvtglosstarget}{Target} system platform differences such as these may incur performance hindrances in that the amount of memory that may be \glslink{dvtglossmemorypagelocking}{locked} could be limited; forcing the solution to perform its bytestream transmission in several instances of \dvttermmagicinstruction s.
Furthermore, \glslink{dvtglossmemorypagelocking}{page locking} functionalities may not be accessible by the user whatsoever, suggesting further studies into how a paravirtualized solution for graphics acceleration may perform using other methodologies for trans-simulation communication; such as TCP/IP-networking (see \dvtcmdcitefur{dissertation:nilsson:2014} for an elaboration on such methodologies).

In section \ref{sec:background_graphicsvirtualization}, potentially costly maintenance of updated graphics frameworks is mentioned as drawback of paravirtualization as a methodology to achieve graphics acceleration.
Later, in section \ref{sec:methodologysolution_openglabigeneration}, it is mentioned that the solution described in this document utilizes software to partly generate paravirtualized graphics \dvttermabi s where possible.
In order to further streamline the maintenance of a productified paravirtualized solution, such framework generation could be automated further by utilizing software such as \texttt{SWIG} or \texttt{SIL} to retrieve function signatures; no longer requiring developers to describe function headers by the means of configuration files as described in section \ref{sec:methodologysolution_openglabigeneration}.

Furthermore, in line with the \dvttermsimics\ attributes described in chapter \ref{cha:background}, a certain set of behavior must be attained in order to align with \dvttermsimics\ philosophy.
Analyses and recommendations surrounding further study of said attributes, in terms of graphics acceleration by the means of paravirtualization, are presented below in paragraphs \dvtcmdrefname{par:appendixa_simicsproductification_deterministicexecution}, \dvtcmdrefname{par:appendixa_simicsproductification_checkpointing}, and \dvtcmdrefname{par:appendixa_simicsproductification_reverseexecution}, respectively.\\

\noindent
Having analyzed the methodology of paravirtualization as a means to achieve graphics acceleration in the \dvttermsimics\ full-system simulator, and making the assumptions presented in paragraphs \dvtcmdrefname{par:appendixa_simicsproductification_deterministicexecution}, \dvtcmdrefname{par:appendixa_simicsproductification_checkpointing}, and \dvtcmdrefname{par:appendixa_simicsproductification_reverseexecution}, there is no obvious obstacle as to why advanced functionalities such as \dvttermdeterministicexecution , \dvttermcheckpointing , or \dvttermreverseexecution\ could not be integrated with paravirtualized graphics acceleration; nor is there any reason to believe that such attributes should severely impair performance of such a solution.

% Deterministic Execution (section Determinism in OpenGL ES)
\paragraph{Deterministic Execution}
\label{par:discussion_deterministicexecution}
\index{Deterministic Execution}
Following an \dvttermapi\ specification such as \dvttermopengles , it naturally follows that there may be implementational differences in-between vendor drivers.
Such variations are commonly insignificant and, although present, do not affect the end-user.
Below, presuming the occurrence of such inconsistencies, a number of scenarios are presented in order to explain how this may affect the desirable traits of the \dvttermsimics\ full-system simulator whilst simulating \dvttermopengles .

In line with driver inconsistencies, there may be cause to believe that rounding of pixel values may vary dependent on \dvttermhost\ driver vendor.
Typically, such variations are far from noticeable and would not offset simulation timing.
Neither would this inconsequential execution affect the local deterministic trait during simulation, as the driver in itself is coherent in its execution.

In \dvttermsimics , determinism and repeatability are key values, meaning that simulation execution must not differ - independent on the \dvttermhost\ system.
For example, one may pose the scenario of \dvttermsimics\ users, using different driver vendors on their respective \dvttermhost\ machines, wishing to share \dvttermcheckpoint s in which an application utilizing \dvttermopengles\ is being executed.
In this manner, posing that the execution paths of said application treads on functionality that is not intricately defined by the \dvttermopengles\ specification, the output may differ.
Note that, presuming the output (usually a buffer to-be presented on-screen) is not calculated upon, this does not affect the \dvttermtiming\ of the users' simulations.
As such, this scenario maintains the determinism and repeatability traits of \dvttermsimics .

However, if this varying output is calculated upon (a plausible scenario would be the compression of a graphics output screenshot, such as the ones presented in section \ref{sec:methodologyexperiment_benchmarking}) the \dvttermtiming\ of simulation is influenced which may propagate - effectively causing a state-change in the simulated \dvttermcpu .
In this manner, the deterministic trait of \dvttermsimics\ would be broken, as certain instructions no longer correspond to their previously \dvttermtiming -accurate cycles.\\

\noindent
More often than not, the scenario described in the above paragraph will not be an issue to \dvttermdeterministicexecution\ in \dvttermsimics , as the framebuffer usually only acts as means to present data to the user.
In this way, the attribute of \dvttermdeterministicexecution\ would not be scathed by integration of paravirtualized graphics in the \dvttermsimics\ full-system simulator presuming the preconditions described above are not compromised.
As such, the incorporation of \dvttermdeterministicexecution\ into the paravirtualized methodology as described in this document is possible through the set of assumptions, as described in this paragraph.

% Checkpointing
\paragraph{Checkpointing}
\label{par:discussion_checkpointing}
\index{Checkpointing}
The incorporation of \dvttermcheckpointrestart\ schemes in the paravirtualized solution described in chapter \ref{cha:methodologysolution} would require an efficient way of saving- and storing the \dvttermopengl\ state, as present in the \dvttermhost\ system.
The possibility of incorporating such functionality has, as mentioned in chapter \ref{cha:relatedwork}, been studied and performed in the \dvttermqemu\ full-system simulator by Lagar-Cavilla et al. in their work on VMGL\dvtcmdcitebib{inproceedings:lagarcavilla:2007}.
Such state retrieval, in the case of \dvttermopenglestwopointo , may be achieved by the \dvtcmdcodeinline{glGet}-set of \dvttermopengl\ library methods\footnote{Note that such fit-for-purpose methods may not necessarily be available in other graphics frameworks, but since this dissertation focuses on \dvttermopenglestwopointo , such dilemmas will not be elaborated upon further in this document.}.
Using these methods, one ought be able to retrieve the entirety of the \dvttermopengl\ state variables present in the \dvttermhost\ system (see \dvtcmdciteref{technicaldocs:silicongraphics:2014} for more information on the \dvtcmdcodeinline{glGet}-family of methods).

Having retrieved these variables, one might subsequently store the changes in \dvttermopengl\ states in some data structure; thus avoiding the need to store the entirety of these states every frame.

As such, albeit circumstantial, there ought be no reason as to why the introduction of \dvttermcheckpoint\ functionality should pose a problem to a paravirtualized graphics framework.\\

\textit{SPECULATION: Had the subject framework of paravirtualization been earlier versions of \dvttermopengl , it might have been possible to use the push- and pop-methodology existent in those frameworks for the solution to know beforehand what states to save.}

% Reverse Execution
\paragraph{Reverse Execution}
\label{par:discussion_reverseexecution}
\index{Reverse Execution}
In the solution devised for the purpose of this study, the paravirtualized solution renders to a window present in the \dvttermhost\ system (see chapter \ref{cha:methodologysolution}).
Effectively, this means that the framebuffer is present in \dvttermhost\ memory.
As such, \dvttermsimics\ \dvttermreverseexecution\ functionality would not include said memory space; leading to the framebuffer presented in in the \dvttermhost\ system not to display in reverse in coagency with the \glslink{dvtglossreverseexecution}{reverse executing} \dvttermtarget\ system.

For a productification of a paravirtualized graphics framework for the \dvttermsimics\ full-system simulator, it is probable that - unlike an academic study into the subject, such as the one presented in this document - the window in which the paravirtualized framework displays its graphics would be located inside of the \dvttermtarget\ system.
If so would be the case, the framebuffer would be located in \dvttermtarget\ memory, rendering said framebuffer eligible for \dvttermreverseexecution\ since such memory is stored in \dvttermcheckpoint s.
As such, assuming the graphics framebuffer being present in \dvttermtarget\ system memory, \dvttermreverseexecution\ would be supported natively by \dvttermsimics .

% Host Virtualization Extensions
\section{Host Virtualization Extensions}
\label{sec:discussion_hostvirtualizationextensions}
For the purposes of the experiment conducted, \dvttermhostvirtualizationextensions\ have been used to accelerate performance of \dvttermxeightysix\ instructions.
Host virtualization extensions reduces much of the overhead incurred by simulating hardware similar to that of the simulation \dvttermhost ; thus stressing the capabilities of the paravirtualized acceleration devised for this study.

However, there are times when utilization of \dvttermhostvirtualizationextensions\ is rendered impossible.
Such examples include simulating systems other than that of the \dvttermhost\ machine - such as \dvttermarm\ architectures - or when a user wishes to set breakpoints in the software stack.
At such times, the simulation is forced to execute without the benefits of native execution; often slowing down the simulation by several orders of magnitude.
It may be argued that hardware assisted graphics acceleration may become particularily influential in such situations, as \dvttermcpu -bound methodologies - such as software rasterization commonly used in \dvttermsimics\ - may incur large performance issues that does not scale well with interpreted instruction sets.

As such, it may be argued that the full capabilities of paravirtualized graphics acceleration, as compared to software rasterization, is not necessarily demonstrated in an environment utilizing \dvttermhostvirtualizationextensions ; a scenario lessening the impact of \dvttermcpu -bound workloads.

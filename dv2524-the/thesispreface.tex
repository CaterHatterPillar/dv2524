% thesispreface.tex

% Preface
\chapter*{Preface}
\label{cha:preface}
\addcontentsline{toc}{chapter}{Preface}

% Acknowledgement
\section*{Acknowledgement}
\label{sec:preface_acknowledgement}
During my employment at \dvttermintel\ I have had the oppertunity to work with an intriguing product employed by a wide array of significant players in the software industry.
I have been met by a creative work environment; being granted exchanges with ambitious collegues in an internationally competitive trade.
As such, the author would like to express his sincere gratitude to \dvttermintel\ and its flourishing work culture.
The author would like to express his gratitude to everyone at the \dvttermintel\ and \dvttermwindriver\ offices in Stockholm, all of whom have been very welcoming during my stay.

The author would like to acknowledge the following persons for their contribution to the project.
These persons are ordered alphabetically in accordance to their surnames\footnote{Note that the order of the persons named is not indicative of the magnitude of their contribution.}.

\begin{itemize}[noitemsep]
	\item Daniel Aarno (\dvttermintel )
	\item Erik Carstensen (\dvttermintel )
	\item Anders Conradi (\dvttermintel )
	\item Mattias Engdegård (\dvttermintel )
	\item Prof. Håkan Grahn (Blekinge Institute of Technology)
	\item Christian Häggström (\dvttermintel )
	\item Stefan Lindblad (\dvttermintel )
	\item Jakob Skoglund (\dvttermwindriver )
	\item Magnus Vesterlund (\dvttermwindriver )
	\item Bengt Werner (\dvttermintel )
\end{itemize}

Furthermore, the author would like to particularily acknowledge the contributions of Erik Carstensen (\dvttermintel ) and Prof. Håkan Grahn (Blekinge Institute of Technology) whom have acted industry- and university advisor, respectively, through the course of this project, in addition to Alexander Mohlin - the acting student opponent to this thesis.

% Structure and Formatting
\section*{Structure and Formatting}
\label{sec:preface_structureandformatting}
This section is presented in order to give the reader an idea how the document is structured and formatted, in order to give the reader an idea what to expect when reading the material presented in this thesis.
Points of interest are summarized in the list below.
\begin{itemize}[noitemsep]
	\item The first occurence of acronyms are written in full. E.g. GNU (GNU's Not Unix!)
	\item The introduction of new terminology is capitalized in its first occurence. E.g. Semiconductor.
	\item The document refrains from using formatting such as italics and bold text in order to increase the readability of the document.
	\item Terminology and acronyms are presented in the List of Terms and List of Acronyms at the back of this dissertation.
	\item Many terms and acronyms used throughout the course of this document are hyperlinked to said compiled lists.
	\item In addition to List of Terms and List of Acronyms, the document is complemented with an Index, which may be found in the dissertation back matter.
\end{itemize}

% Industry Collaboration
\section*{Industry Collaboration}
\label{sec:preface_industrycollaboration}
This study was performed at the \dvttermintel\ offices in Stockholm, in collaboration with \dvttermintel\ and \dvttermwindriver\
During this time, the author was employed at \dvttermintel\ and performed the majority of this thesis as part of his duties there.

% Accessibility
\section*{Accessibility}
\label{sec:preface_accessibility}
This document, along with its source code and revision history, is hosted as an open source repository with the entirety of the raw data collected for the purposes of this study.
As such, this data may be reviewed and analysed further.

Furthermore, the process of releasing the benchmarking source code under an open source license is underway.
References to said source code may be found in latter revisions of this document.

In the back of this document, a unique commit hash is enclosed identifying the corresponding Git commit and its performed revisions.
Later revisions of this document, along with other resources used for the purpose of this study, may be found at:
\begin{center}
\href{https://github.com/CaterHatterPillar/dv2524}{www.github.com/CaterHatterPillar/dv2524}
\end{center}

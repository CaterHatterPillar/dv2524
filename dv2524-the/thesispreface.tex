% thesispreface.tex

% Preface
\chapter*{Preface}
\label{cha:preface}
\addcontentsline{toc}{chapter}{Preface}

% Acknowledgement
\section*{Acknowledgement}
\label{sec:preface_acknowledgement}
During my employment at \dvttermintel\ I have had the opportunity to work with an intriguing product employed by a wide array of significant players in the software industry.
I have been met by a creative work environment; being granted exchanges with ambitious colleagues in an internationally competitive trade.
As such, I would like to express my sincere gratitude to \dvttermintel\ and its flourishing work culture.
Furthermore, I would like to thank everyone at the \dvttermintel\ and \dvttermwindriver\ offices in Stockholm, all of whom have been very welcoming during my stay.

In particular, the following persons ought be acknowledged for their contribution to the project.
These persons are ordered alphabetically in accordance to their surnames\footnote{Note that the order of the persons named is not indicative of the magnitude of their contribution.}.

\begin{itemize}[noitemsep]
	\item Daniel Aarno (\dvttermintel )
	\item Erik Carstensen (\dvttermintel )
	\item Anders Conradi (\dvttermintel )
	\item Mattias Engdegård (\dvttermintel )
	\item Prof. Håkan Grahn (Blekinge Institute of Technology)
	\item Christian Häggström (\dvttermintel )
	\item Stefan Lindblad (\dvttermintel )
	\item Jakob Skoglund (\dvttermwindriver )
	\item Magnus Vesterlund (\dvttermwindriver )
	\item Bengt Werner (\dvttermintel )
\end{itemize}

Additionally, I would like to acknowledge the contributions of Erik Carstensen (\dvttermintel ) and Prof. Håkan Grahn (Blekinge Institute of Technology) whom have acted industry- and university advisor, respectively, throughout the course of this project, in addition to Alexander Mohlin - the acting student opponent to this thesis.

% Structure and Formatting
\section*{Structure and Formatting}
\label{sec:preface_structureandformatting}
This section is presented to give the reader an idea of how the document is structured and formatted, and what to expect when reading the material presented in this thesis.
Points of interest are summarized in the list below.
\begin{itemize}[noitemsep]
	\item The first occurrence of acronyms are written in full, e.g. GNU (GNU's Not Unix!).
	\item Some acronyms are not expanded, such as \dvttermcpu . This is due to said acronyms having become standardized in their usage. As such, the reader is expected to be familiarized with the acronyms in question. These are still featured in the List of Acronyms, however. 
	\item The introduction of new terminology is capitalized in its first occurrence, e.g. Semiconductor.
	\item The document refrains from using formatting such as italics and bold text in order to increase the readability of the document.
	\item Terminology and acronyms are presented in the List of Terms and List of Acronyms in the back matter of this dissertation.
	\item Many terms and acronyms used throughout the course of this document are hyperlinked to said compiled lists.
	\item Square brackets are used solely for citations.
	\item For the purposes of discerning complementing text from references, nestled parentheses are not distinguished using any other formatting.
	\item Cross-references to body matter divisions (such as chapter, section, or paragraph) are purposely non-capitalized, as to increase document readability.
	\item For the sake of visualization, values outside of corresponding standard deviation are not presented in graphs unless specified otherwise.
\end{itemize}

% Industry Collaboration
\section*{Industry Collaboration}
\label{sec:preface_industrycollaboration}
This study was performed at the \dvttermintel\ offices in Stockholm, in collaboration with \dvttermintel\ and \dvttermwindriver\
During this time, the author was employed at \dvttermintel\ and performed the majority of this thesis as part of his duties there.

% Accessibility
\section*{Accessibility}
\label{sec:preface_accessibility}
This document, along with its source code and revision history, is hosted as an open source repository which includes the entirety of the raw data collected for the study.
As such, this data may be reviewed and analyzed for future studies.
Furthermore, the process of releasing the benchmarking source code, under an open source license, is underway at the \dvttermintel\ Open Source Center.
References to said source code may be found in latter revisions of this document.

In the thesis back matter, a unique commit hash is enclosed identifying the corresponding Git commit and its changes.
The Git repository, along with supplementary material produced during the study, may be found at:
\begin{center}
\href{https://github.com/CaterHatterPillar/dv2524}{www.github.com/CaterHatterPillar/dv2524}
\end{center}

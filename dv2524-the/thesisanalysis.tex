% thesisanalysis.tex
% Chapter Analysis.

% TODO:
% Strengths of solution.
% Identified bottlenecks.
% Magic instruction profiling (point out that, although tests show no larger impact of just a magic instruction with nops, that magic instruction require to exit hardware aided virtualization, JIT-compilation optimization, and to simply interpret the occurence of a magic instruction. Speculate surrounding how this may affect the outcome)
% Speculation surrounding memory page translation.

% Analysis
\chapter{Analysis}
\label{cha:analysis}
As expanded upon in \dvtcmdcitefur{dissertation:nilsson:2014}, there are a number of ways of virtualizing \dvttermgpu s in system simulators, a few of which accomodate for hardware acceleration of \dvttermgpu\ kernels.
When faced with tackling the issue of \dvttermgpu\ virtualization, there are equally many variables to consider as there are options; the first of which is the purpose of said virtualization.
The \dvttermsimics\ architectural simulator is by all means a full-system simulator; meaning, as portrayed in section \dvtcmdrefname{sec:appendixa_simics} under \dvtcmdrefname{cha:appendixa}, that it may run complete real-software stacks without modification.
However, \dvttermsimics\ is intended to feature a low level of timing fidelity for the purposes of high performance, and is - as such - not a cycle-accurate simulator\footnote{This should not be confused with...}\todo{?}.
As such, and in line with the considerations for \dvttermgpu\ virtualization - one must analyze and balance the purposes of simulation, as there is not always a winning general-case.
In this way, methodologies with varying levels of implementational accuracy present themselves - from slow low-level instruction set modeling to fast high level paravirtualization.
Said methods are presented in the paragraphs below.

% GPU Modeling
\paragraph{GPU Modeling}
\label{par:analysis_gpumodeling}
Firstly, one may consider developing a full-fletched \dvttermgpu\ model; that is, implementing virtualization of the \dvttermgpu\ \dvttermisa .
This methodology may be appropriate for the purposes of low-level development close to \dvttermgpu\ hardware.
For example, one might imagine the scenario of driver development for next-generation \dvttermgpu s, as portrayed in the thesis proposal (see \dvtcmdcitefur{dissertation:nilsson:2014}).

However, the development of \dvttermgpu\ models, similar to that of common architectural model development for the \dvttermsimics\ full-system simulator, incurs a number of flaws.
The first of these flaws, due to \dvttermgpu\ hardware - still - often being poorly documented\todo{citation needed!}, on the contrary to \dvttermcpu\ architectures, driving estimated development costs to unsustainable levels.
Furthermore, such modeling of massively parallelized \dvttermgpu\ technology on \dvttermcpu s induce high costs rendering the methodology less preferable for development requiring some application speed.

% PCI Passthrough
\paragraph{PCI Passthrough}
\label{par:analysis_pcipassthrough}
Secondly, one ought consider the benefits of \dvttermpcipassthrough ; allowing virtual systems first-hand - exclusive - access to \dvttermhost\ machine devices\dvtcmdciteref{web:jones:2009}.
The direct contact with \dvttermhost\ system devices accomodated by methodologies such as \dvttermpcipassthrough\ enable performance, hardware accelerated, graphics acceleration.

Yet, the methodology suffers from several disadvantages, such as only being capable of running on \dvttermlinux\ systems \todo{citation needed!}.
Additionally, the solution requires dedicated hardware, followed by the \dvttermhost\ system losing all access to said devices during the course of simulation.
In terms of \dvttermgpu\ virtualization, this would induce the necessity of the \dvttermhost\ machine featuring multiple graphics cards.
The tight coupling induced by direct contact with \dvttermhost\ hardware also requires the simulation \dvttermtarget\ to utilize the same device drivers as the \dvttermhost\ system, rendering the methodology unflexible in term so \dvttermgpu\ virtualization diversity.
In line with a paravirtualized approach, \dvttermpcipassthrough\ also requires modification of the \dvttermtarget\ system - in addition to configuration of the simulation \dvttermhost .

% Linux only

% Soft Modeling
\paragraph{Soft Modeling}
\label{par:analysis_softmodeling}
\ldots

% TODO:
% DV2549

% Paravirtualization
\paragraph{Paravirtualization}
\label{par:analysis_paravirtualization}
\ldots

% TODO:
% Why paravirtualization is the sensible choice.

% TODO: Insert figure visualizing virtualization methodologies (see https://github.com/CaterHatterPillar/dv2524/issues/161).
\missingfigure[figwidth=6cm]{Visualization of GPU virtualization methodologies.}

% Benchmark variations
\section{Benchmark Variations}
\label{sec:analysis_benchmarkvariations}
\ldots

% TODO:
% Expand upon Phong deviations.
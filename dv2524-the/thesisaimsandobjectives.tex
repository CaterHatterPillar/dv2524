% thesisaimsandobjectives.tex
% Chapter Aims & Objectives.

% Aims & Objectives
\chapter{Aims \& Objectives}
\label{cha:aimsandobjectives}
The scope of the study outlined in \dvtcmdcitefur{dissertation:nilsson:2014} consist of implementing paravirtualization of a graphics \dvttermapi\ (being \dvttermopenglestwopointo ) in the \dvttermsimics\ full system simulator developed by \dvttermintel\ and sold through \dvttermintel 's subsidiary \dvttermwindriver\ 
Said implementation adheres to a certain \dvttermreferenceimplementation , which is elaborated upon in section \ref{sec:relatedwork_qemu}.

As such, this study concerns investigating the performance, and the feasability of extended benefits and advanced functionality, of paravirtualized graphics acceleration in a virtual platform.
Said integration entails investigating, analyzing, and developing methods and techniques for efficient communication and execution in the \dvttermsimics\ run-time environment, in addition to identifying liabilities of paravirtualized technologies in regards to \dvttermsimics\ philosophy (being high-performance and \dvttermtiming -accurate determinism and repeatability\dvtcmdcitebib{journals:aarno:2013}).
As such, this dissertation does not exclusively concern \dvttermsimics\ integration, but an investigation into paravirtualized drivers in virtual platforms.

Accordingly, for the purpose of this thesis, a paravirtualized solution for graphics acceleration in \dvttermsimics\ has been developed to accomodate for this analysis.
Said solution is the subject of the thesis presented in this document.

Based off the solution developed for the purpose of this study, this document comprises an analysis into the performance of said solution compared to that of traditional software rasterization.
As such, the objectives of this dissertation is to establish and present the method of paravirtualization as a feasable way of accelerating graphics in virtual platforms.

Furthermore, and presented in chapter \ref{cha:futurework}, this thesis also comprise an analysis into the prerequisites for achieving the valued attributes present in the \dvttermsimics\ full-system simulator, as elaborated upon in chapter \ref{cha:background}, in terms of paravirtualized graphics.

% thesisterms.tex
% Defines commonly used phrases to save time on common formatting.
% ---
% | Alphabetical
% | Plain text term (grep magic)
% | Glossary entry / Acronym entry
% | Command definition

% A
% Application Programming Interface
\newacronym{dvtglossapi}{API}{Application Programming Interface}
\newcommand{\dvttermapi}{\dvtcmdabbrev{dvtglossapi}}
% Application Binary Interface
\newacronym{dvtglossabi}{ABI}{Application Binary Interface}
\newcommand{\dvttermabi}{\dvtcmdabbrev{dvtglossabi}}
% Android
\newcommand{\dvttermandroid}{Android}
% Android SDK
\newcommand{\dvttermandroidsdk}{\dvttermandroid\ SDK}
% Android Emulator
\newcommand{\dvttermandroidemulator}{\dvttermandroid\ emulator}
% ACM
\newcommand{\dvttermacm}{ACM}
% ARM
\newcommand{\dvttermarm}{ARM}

% B
% ...

% C
% Checkpoint
\newglossaryentry{dvtglosscheckpoint}
{
  name=Checkpoint,
  description={is a saved state of things. See also \dvttermcheckpointing .},
  plural=Checkpoints
}
\newcommand{\dvttermcheckpoint}{\dvtcmdcaponce{Checkpoint}}
% Checkpointing
\newglossaryentry{dvtglosscheckpointing}
{
  name=Checkpointing,
  description={See section \ref{sec:simics_checkpointing}}
}
\newcommand{\dvttermcheckpointing}{\dvtcmdcaponcegloss{dvtglosscheckpointing}{Checkpointing}}
% Central Processing Unit
\newacronym{dvtglosscpu}{CPU}{Central Processing Unit}
\newcommand{\dvttermcpu}{\glslink{dvtglosscpu}{CPU}} % Assume the reader knows about CPUs, so don't print the acronym in plain text the first time it is mentioned.
% Checkpoint / Restart
\newglossaryentry{dvtglosscheckpointrestart}
{
  name=Checkpoint / Restart,
  description={Description of Checkpoint / Restart.}
}
\newcommand{\dvttermcheckpointrestart}{\dvtcmdcaponcegloss{dvtglosscheckpointrestart}{Checkpoint / Restart}}
\newcommand{\dvttermcuda}{CUDA}
% C (Programming language)
\newcommand{\dvttermc}{\texttt{C}}
% C++
\newcommand{\dvttermcplusplus}{\texttt{C++}}
% Cisco
\newcommand{\dvttermcisco}{Cisco Systems, Inc.}


% D
% Direct Virtualization
\newglossaryentry{dvtglossdirectvirtualization}
{
  name=Hardware-assisted Virtualization,
  description={A platform virtualization approach that enables efficient full virtualization using help from hardware capabilities, primarily from host processors. Also known as accelerated virtualization}
}
\newcommand{\dvttermdirectvirtualization}{\dvtcmdcaponcegloss{dvtglossdirectvirtualization}{Hardware-assisted Virtualization}}
% DirectX
\newcommand{\dvttermdirectx}{DirectX}
% Deterministic Execution
\newglossaryentry{dvtglossdeterministicexecution}
{
  name=Deterministic Execution,
  description={See section \nameref{sec:simics_deterministicexecution} under chapter \nameref{cha:appendixa}}
}
\newcommand{\dvttermdeterministicexecution}{\dvtcmdcaponcegloss{dvtglossdeterministicexecution}{Deterministic Execution}}
% Deterministic Algorithm
\newglossaryentry{dvtglossdeterministicalgorithm}
{
  name=\href{http://en.wikipedia.org/wiki/Deterministic_algorithm}{Deterministic Algorithm},
  description={An algorithm which, given a particular input, will always produce the same output, with the underlying machine always passing through the same sequence of states},
  sort={Deterministic Algorithm},
}
\newcommand{\dvttermdeterministicalgorithm}{\dvtcmdcaponcegloss{dvtglossdeterministicalgorithm}{Deterministic Algorithm}}

% E
% Ericsson
\newcommand{\dvttermericsson}{Ericsson}
% EGL
\newglossaryentry{dvtglossegl}
{
  name={\href{http://en.wikipedia.org/wiki/EGL_(API)}{EGL}},
  description={EGL is an interface between Khronos rendering APIs (such as OpenGL, OpenGL ES or OpenVG) and the underlying native platform windowing system},
  sort={EGL},
}
\newcommand{\dvttermegl}{\glslink{dvtglossegl}{EGL}}

% F
% Fedora
\newglossaryentry{dvtglossfedora}
{
    name={\href{http://en.wikipedia.org/wiki/Fedora_(operating_system)}{Fedora (operating system)}},
    description={An operating system based on the Linux kernel, developed by the community-supported Fedora Project and owned by Red Hat},
    sort={Fedora},
}
\newcommand{\dvttermfedora}{\glslink{dvtglossfedora}{Fedora 19}}
% Full-system simulation
\newacronym{dvtglossfss}{FSS}{Full-system Simulation}
\newglossaryentry{dvtglossfullsystemsimulation}
{
    name={\href{http://en.wikipedia.org/wiki/Full_system_simulator}{Full-system simulation}},
    description={Full-system simulation denotes an architecture simulator that simulates an electronic system at such a level of detail that complete software stacks from real systems can run on the simulator without any modification},
    sort={Full-system simulation},
}
\newcommand{\dvttermfullsystemsimulation}{\dvtcmdcaponcegloss{dvtglossfullsystemsimulation}{Full-system Simulation}}
% Freescale Semiconductor
\newcommand{\dvttermfreescalesemiconductor}{Freescale Semiconductor, Inc.}
% Frames Per Second
\newacronym{dvtglossfps}{FPS}{Frames Per Second}
\newcommand{\dvttermfps}{\dvtcmdabbrev{dvtglossfps}}
% Floating Point Operations Per Second
\newacronym{dvtglossflops}{FLOPS}{Floating-point Operations Per Second}
\newcommand{\dvttermflops}{\dvtcmdabbrev{dvtglossflops}}

% G
% Graphics Processing Unit
\newacronym{dvtglossgpu}{GPU}{Graphics Processing Unit}
\newcommand{\dvttermgpu}{\dvtcmdabbrev{dvtglossgpu}}
% General Purpose computing on Graphics Processing Units
\newacronym{dvtglossgpgpu}{GPGPU}{General Purpose computing on Graphics Processing Units}
\newcommand{\dvttermgpgpu}{\dvtcmdabbrev{dvtglossgpgpu}}
% Google
\newcommand{\dvttermgoogle}{Google}
% GE Avionics
\newcommand{\dvttermgeavionics}{GE Aviation}
% Graphical User Interface
\newacronym{dvtglossgui}{GUI}{Graphical User Interface}
\newcommand{\dvttermgui}{\dvtcmdabbrev{dvtglossgui}}

% H
% Host
\newglossaryentry{dvtglosshost}
{
  name={Simulation Host},
  description={Description of a Simulation Host.}
}
\newcommand{\dvttermhost}{\dvtcmdcaponcegloss{dvtglosshost}{Host}}
% Hewlett-Packard
\newcommand{\dvttermhewlettpackard}{Hewlett-Packard Company}
% Honeywell
\newcommand{\dvttermhoneywell}{Honeywell International, Inc.}
% Hypervirtualization
\newglossaryentry{dvtglosshypersimulation}
{
  name={Hypersimulation},
  description={Technologies making a simulated processor skip through, what would effectively be, idle time rather than executing \dvtcmdcodeinline{nop}- type instructions cycle by cycle},
}
\newcommand{\dvttermhypersimulation}{\dvtcmdcaponcegloss{dvtglosshypersimulation}{Hypersimulation}}
% Host Virtualization Extentions
\newglossaryentry{dvtglosshostvirtualizationextensions}
{
  name={\href{http://en.wikipedia.org/wiki/Hardware-assisted_virtualization}{Host virtualization extensions}},
  description={A platform virtualization approach that enables efficient full virtualization using help from hardware capabilities, primarily from the host processors. Also known as hardware-assisted virtualization},
  sort={Host virtualization extensions},
}
\newcommand{\dvttermhostvirtualizationextensions}{\dvtcmdcaponcegloss{dvtglosshostvirtualizationextensions}{Host Virtualization Extensions}}
\newacronym{dvtglosshaxm}{HAXM}{Hardware Accelerated Execution Manager}
\newcommand{\dvttermhaxm}{\dvttermintel\ \dvtcmdabbrev{dvtglosshaxm}}

% I
% Interpretation
\newcommand{\dvtterminterpretation}{\dvtcmdcaponce{Interpretation}}
% Intel
\newcommand{\dvttermintel}{Intel\circledR}
% Intel Core i7
\newcommand{\dvttermintelcoreiseven}{\dvttermintel\ Core\texttrademark\ i7}
% IBM
\newcommand{\dvttermibm}{IBM}
% Instruction Set Architecture
\newacronym{dvtglossisa}{ISA}{Instruction Set Architecture}
\newcommand{\dvttermisa}{\dvtcmdabbrev{dvtglossisa}}
% Institute of Electrical and Electronics Engineers (IEEE)
\newacronym{dvtglossieee}{IEEE}{Institute of Electrical and Electronics Engineers}
\newcommand{\dvttermieee}{\dvtcmdabbrev{dvtglossieee}}
% IEEE Standard for Floating-Point Arithmetic (IEEE 754)
\newglossaryentry{dvtglossieeefp}
{
  name={IEEE 754},
  description={IEEE Standard for Floating-Point Arithmetic (IEEE 754) is a technical standard for floating-point computation established in 1985 by the \dvttermieee },
  sort={IEEE Floating Point},
}
\newcommand{\dvttermieeefp}{\glslink{dvtglossieeefp}{IEEE 754}}
% iOS
\newcommand{\dvttermios}{iOS}

% J
% Just-in-Time
\newacronym{dvtglossjit}{JIT}{Just-in-Time}
\newcommand{\dvttermjit}{\dvtcmdabbrev{dvtglossjit}}
% Java Native Interface
\newacronym{dvtglossjni}{JNI}{Java Native Interface}
\newcommand{\dvttermjni}{\dvtcmdabbrev{dvtglossjni}}

% K
% Khronos
\newglossaryentry{dvtglosskhronos}
{
  name={\href{http://en.wikipedia.org/wiki/Khronos_Group}{Khronos Group}},
  description={The Khronos Group is a member-funded industry consortium focused on the creation of open standard \dvttermapi s on a wide variety of platforms},
  sort={Khronos Group},
}
\newcommand{\dvttermkhronos}{\glslink{dvtglosskhronos}{Khronos}}
% Kernel-based Virtual Machine
\newacronym{dvtglosskvm}{KVM}{Kernel-based Virtual Machine}
\newcommand{\dvttermkvm}{\dvtcmdabbrev{dvtglosskvm}}

% L
\newcommand{\dvttermlinux}{Linux}
\newcommand{\dvttermlockheedmartin}{Lockheed Martin}

% M
% Million Instructions Per Second
\newacronym{dvtglossmipssecond}{MIPS}{Million Instructions Per Second}
\newcommand{\dvttermmipsecond}{\dvtcmdabbrev{dvtglossmipssecond}}
% Microsoft
\newcommand{\dvttermmicrosoft}{Microsoft}
% Magic Instruction
\newglossaryentry{dvtglossmagicinstruction}
{
  name={Magic Instruction},
  description={See section \dvtcmdrefname{sec:backgroundandrelatedwork_magicinstructions} under \dvtcmdrefname{cha:appendixa}},
}
\newcommand{\dvttermmagicinstruction}{\dvtcmdcaponcegloss{dvtglossmagicinstruction}{Magic Instruction}}
% Memory Management Unit (MMU)
\newacronym{dvtglossmmu}{MMU}{Memory Management Unit}
\newcommand{\dvttermmmu}{\dvtcmdabbrev{dvtglossmmu}}
% Memory Page Locking
\newglossaryentry{dvtglossmemorypagelocking}
{
  name={Memory Page Locking},
  description={Some \dvttermos s may force memory pages not to be swapped to secondary storage, referred to as 'pinned', 'locked', 'fixed', or 'wired' memory pages},
}

% N
% NASA
\newcommand{\dvttermnasa}{NASA}
\newcommand{\dvttermnortel}{Nortel Networks Corporation}
\newcommand{\dvttermnorthropgrumman}{Northrop Grumman Corporation}

% O
% OpenGL
\newcommand{\dvttermopengl}{OpenGL}
% OpenGL ES
\newcommand{\dvttermopengles}{\dvttermopengl ~ES}
% OpenGL ES 2.0
\newcommand{\dvttermopenglestwopointo}{\dvttermopengles ~\dvtcmdnum{2.0}}
% Operating System
\newacronym{dvtglossos}{OS}{Operating System}
\newcommand{\dvttermos}{\dvtcmdabbrev{dvtglossos}}

% P
% Peripheral Component Interconnect
\newacronym{dvtglosspci}{PCI}{Peripheral Component Interconnect}
\newcommand{\dvttermpci}{\dvtcmdabbrev{dvtglosspci}}
% PCI Passthrough
\newglossaryentry{dvtglosspcipassthrough}
{
  name={PCI passthrough},
  description={PCI passthrough may allow a simulation target exclusive control of physical PCI devices, such as GPUs, audio controllers, and USB controllers},
  sort={PCI Passthrough},
}
\newcommand{\dvttermpcipassthrough}{\dvttermpci\ \dvtcmdcaponcegloss{dvtglosspcipassthrough}{Passthrough}}
% Paravirtualization
\newglossaryentry{dvtglossparavirtualization}
{
  name=\href{http://en.wikipedia.org/wiki/Paravirtualization}{Paravirtualization},
  description={A virtualization technique that presents a software interface to virtual machines that is similar, but not identical to that of the underlying hardware},
  sort={Paravirtualization},
}
\newcommand{\dvttermparavirtualization}{\dvtcmdcaponcegloss{dvtglossparavirtualization}{Paravirtualization}}
% Python
\newcommand{\dvttermpython}{\texttt{Python}}

% Q
% QEMU
\newglossaryentry{dvtglossqemu}
{
  name=\href{http://en.wikipedia.org/wiki/QEMU}{QEMU},
  description={A free and open-source hosted hypervisor that performs hardware virtualization},
  sort={QEMU},
}
\newcommand{\dvttermqemu}{\glslink{dvtglossqemu}{QEMU}} % QEMU is an abbreviation, but is not formatted as such since it is also a product name.

% R
% Reverse Execution
\newglossaryentry{dvtglossreverseexecution}
{
  name=Reverse Execution,
  description={See section \ref{sec:simics_reverseexecution}}
}
\newcommand{\dvttermreverseexecution}{\dvtcmdcaponcegloss{dvtglossreverseexecution}{Reverse Execution}}
% Reference Solution
\newglossaryentry{dvtglossreferencesolution}
{
  name=Reference Solution,
  description={See section \ref{sec:backgroundandrelatedwork_qemu}}
}
\newcommand{\dvttermreferencesolution}{\dvtcmdcaponcegloss{dvtglossreferencesolution}{Reference Solution}}

% S
% Software Rendering
\newcommand{\dvttermsoftwarerendering}{\dvtcmdcaponce{Software Rendering}}
% Simics
\newglossaryentry{dvtglosssimics}
{
  name=The Simics full-system simulator,
  description={See chapter \ref{cha:simics}}
}
\newcommand{\dvttermsimics}{\glslink{dvtglosssimics}{Simics}}
% SICS
\newacronym{dvtglosssics}{SICS}{Swedish Institute of Computer Science}
\newcommand{\dvttermsics}{\dvtcmdabbrev{dvtglosssics}}
% Sun Microsystems
\newcommand{\dvttermsunmicrosystems}{Sun Microsystems, Inc.}
% Single Instruction Multiple Data (SIMD)
\newacronym{dvtglosssimd}{SIMD}{Single Instruction, Multiple Data}
\newcommand{\dvttermsimd}{\dvtcmdabbrev{dvtglosssimd}}

% T
% Time-to-Market
\newacronym{dvtglossttm}{TTM}{Time-to-Market}
\newcommand{\dvttermttm}{\dvtcmdabbrev{dvtglossttm}}
% Timing
\newglossaryentry{dvtglosstiming}
{
  name=Simulation Timing,
  description={Description of Timing.}
}
\newcommand{\dvttermtiming}{\dvtcmdcaponcegloss{dvtglosstiming}{Timing}}
% Target
\newglossaryentry{dvtglosstarget}
{
  name=Simulation Target,
  description={Description of Target.}
}
\newcommand{\dvttermtarget}{\dvtcmdcaponcegloss{dvtglosstarget}{Target}}

% U
\newacronym{dvtglossuart}{UART}{Universal Asynchronous Reciever/Transmitter}
\newcommand{\dvttermuart}{\dvtcmdabbrev{dvtglossuart}}

% V
% Virtutech
\newcommand{\dvttermvirtutech}{Virtutech}
% VMware, Inc.
\newcommand{\dvttermvmware}{VMware, Inc.}
% Valve Corporation
\newcommand{\dvttermvalve}{Valve Corporation}

% W
% WARP
\newacronym{dvtglosswarp}{WARP}{Windows Advanced Rasterization Platform}
% Microsoft WARP
\newcommand{\dvttermwarp}{\dvttermmicrosoft\ \dvtcmdabbrev{dvtglosswarp}}
% Wind River Systems, Inc.
\newcommand{\dvttermwindriver}{Wind River Systems, Inc.}
% Windows
\newcommand{\dvttermwindows}{Windows}

% X
% X11
\newglossaryentry{dvtglossxeleven}
{
  name=\href{http://en.wikipedia.org/wiki/X_Window_System}{X Window System},
  description={A windowing system for bitmap displays, common on UNIX-like operating systems},
  sort={X Window System}, % Must add sorting attribute, since sort will otherwize erronously sort by href-command.
}
\newcommand{\dvttermxeleven}{\glslink{dvtglossxeleven}{X11}}
% x86
\newcommand{\dvttermxeightysix}{x86}

% Y
% ...
% Z
% ...

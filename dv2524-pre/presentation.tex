% presentation.tex

\documentclass{beamer}
\usetheme{Intel}

% Document configuration
%\setbeameroption{show notes} % If annotations are desired.
\setbeamertemplate{section in toc}{\inserttocsectionnumber.~\inserttocsection}
\setbeamertemplate{subsection in toc}{\hspace{0.5cm}\rule[0.3ex]{3pt}{3pt}~\inserttocsubsection\par}
\AtBeginSection[]
{
	\begin{frame}
		\frametitle{Table of Contents}
		\begin{multicols}{2}
			\tableofcontents[currentsection]
		\end{multicols}
    \end{frame}
}

% Package inclusion:
\usepackage{tikz} % Used by presentationprogressbar.tex
\usepackage{hyperref} % Used to invoke Santa Claus.
\usepackage{multicol} % Used to split up table of contents in a double column style.

% Package configuration:
\hypersetup{
	colorlinks=true,
	linkcolor=blue,
	urlcolor=blue,
	citecolor=blue,
	anchorcolor=blue
}
\usetikzlibrary{calc}

% presentationprogressbar.tex
% Defines a progress bar used at the top of each frame.
% ---
% Written by Gonzalo Medina (http://tex.stackexchange.com/users/3954/gonzalo-medina).
% http://tex.stackexchange.com/questions/59742/progress-bar-for-latex-beamer
% Jun 13 '12
% edited Jun 21 '12

\definecolor{pbblue}{HTML}{0A75A8}% color for the progress bar and the circle 

\makeatletter
\def\progressbar@progressbar{} % the progress bar
\newcount\progressbar@tmpcounta% auxiliary counter
\newcount\progressbar@tmpcountb% auxiliary counter
\newdimen\progressbar@pbht %progressbar height
\newdimen\progressbar@pbwd %progressbar width
\newdimen\progressbar@rcircle % radius for the circle
\newdimen\progressbar@tmpdim % auxiliary dimension

\progressbar@pbwd=\linewidth
\progressbar@pbht=1pt
\progressbar@rcircle=2.5pt

% the progress bar
\def\progressbar@progressbar{%

    \progressbar@tmpcounta=\insertframenumber
    \progressbar@tmpcountb=\inserttotalframenumber
    \progressbar@tmpdim=\progressbar@pbwd
    \multiply\progressbar@tmpdim by \progressbar@tmpcounta
    \divide\progressbar@tmpdim by \progressbar@tmpcountb

  \begin{tikzpicture}
    \draw[pbblue!30,line width=\progressbar@pbht]
      (0pt, 0pt) -- ++ (\progressbar@pbwd,0pt);

    \filldraw[pbblue!30] %
      (\the\dimexpr\progressbar@tmpdim-\progressbar@rcircle\relax, .5\progressbar@pbht) circle (\progressbar@rcircle);

    \node[draw=pbblue!30,text width=3.5em,align=center,inner sep=1pt,
      text=pbblue!70,anchor=east] at (0,0) {\insertframenumber/\inserttotalframenumber};
  \end{tikzpicture}%
}

\addtobeamertemplate{headline}{}
{%
  \begin{beamercolorbox}[wd=\paperwidth,ht=4ex,center,dp=1ex]{white}%
    \progressbar@progressbar%
  \end{beamercolorbox}%
}
\makeatother

\title{Paravirtualizing OpenGL ES in Simics}
\subtitle{Master's Thesis in Computer Science}
\author{Eric Nilsson}
\institute{Blekinge Institute of Technology}
\date{\today} % Fix

% Opposition presentation:
% 20 minutes
% Three key points:
% * Paravirtualization is feasible for accelerating graphics in virtual platforms.
% * Magic instructions is good methodology to transmit real-time invocations between target and host systems.
% * Paravirtualization may be subject to advanced functionality.
% The frames should lead to these conclusions.

% Screen-play:
% ---
% Front page
% Introduction:
% * (Virtualization)
% * Simics
% * [SHOW VIDEO]
% * Present key points
% Background:
% * Graphics virtualization
% Paravirtualization:
% * Solution architecture
% * (Man-in-the-middle windows)
% * Magic instructions and memory table traversal
% Experiment:
% * Benchmarks
% * Profiling methodology
% Results:
% * Chess results
% * Julia results
% * (Phong results)
% Conclusion:
% * Magic instruction overhead
% * Performance gains
% * Key points recap
% Acknowledgement
% Back page
% ---

\begin{document}
	% FRONT MATTER
	% ---
	% presentationfront.tex

\begin{frame}
	\titlepage
\end{frame}
	
	% BODY MATTER
	% ---
	% INTRODUCTION
	\section{Introduction}
	% Simics
	\subsection{Simics}
	% presentationsimics.tex

\begin{frame}%[allowframebreaks]

\frametitle{Wind River\texttrademark\ Simics\texttrademark }

\begin{itemize}
	\item Full-system simulator\note{Meaning an architectural simulator which may run an unmodified software stack.}
	\item Originally devised at SICS\footnote{The Swedish Institute of Computer Science.}\note{This was the first instance of an unmodifed OS running in an entirely simulated environment.}
	\item Developed by Intel\textregistered 
	\item Sold through Intels subsidiary Wind River Systems, Inc.
	\item Used in the industry by groups such as:
	\begin{itemize}
		\item IBM
		\item NASA
		\item Lockheed Martin
	\end{itemize}
	\item Utilized extensively in academia\footnote{$300+$ universities.}
\end{itemize}

%\framebreak 

%\begin{itemize}
%	\item Deterministic Execution
%	\item Checkpointing
%	\item Reverse Execution
%\end{itemize}

\end{frame}

	% Demonstration
	\subsection{Demonstration}
	% presentationsimics.tex

\begin{frame}	

\frametitle{Demonstration}

Julia Benchmark
\begin{itemize}
	\item \href{http://youtube.com/embed/GKs6OlWKFV8?rel=0&vq=hd1080&autoplay=1}{[Hardware accelerated on the simulation host]}
	\item \href{http://youtube.com/embed/3sCyzppFL0w?rel=0&vq=hd1080&autoplay=1}{[Software rasterized on the simulation target]}
	\item \href{http://youtube.com/embed/__d_EeZBzwc?rel=0&vq=hd1080&autoplay=1}{[Paravirtualized on the simulation target]}
\end{itemize}

% TODO:
% Complement with performance approximation table

\end{frame}
	% Key points
	\subsection{Key points}
	% presentationkeypoints.tex

\begin{frame}
\frametitle{Key points}

\begin{thm}<1->
	Paravirtualization is feasible for accelerating graphics in virtual platforms.
\end{thm}

\begin{thm}<1->
	Magic instructions is promising methodology to carry real-time invocations between target and host systems.
\end{thm}

\begin{thm}<1->
	Paravirtualization may be subject to deterministic execution, checkpointing, and reverse execution.
\end{thm}

\end{frame}


	% BACKGROUND
	\section{Background}
	% Graphics virtualization
	\subsection{Graphics virtualization}
	% presentationgraphicsvirtualization.tex

\begin{frame}
\frametitle{Graphics virtualization}

\begin{columns}
	\column{0.5\textwidth}
	\begin{block}{GPU modeling}
		Develop a GPU model virtualizing the GPU Instruction Set Architecture (ISA).
	\end{block}
	\begin{block}{PCI passthrough}
		Utilize passthrough methodology; granting virtual systems first-hand access to host machine devices.
	\end{block}
    \column{0.5\textwidth}
    \begin{block}{Soft modeling}
    	Use advanced software rasterizers optimizing GPU kernel simulation for CPU architectures.
    \end{block}
    \begin{block}{Paravirtualization}
    	Selectively modify the virtual architecture to accomodate scalability, performance, and simplicity.
    \end{block}
\end{columns}
	
\end{frame}

	% PARAVIRTUALIZATION
	\section{Paravirtualization}

	\section{test}
	\subsection{test}
	\section{test}
	\subsection{test}
	\section{test}
	\subsection{test}
	\section{test}
	\subsection{test}
	\section{test}
	\subsection{test}
	\section{test}
	\subsection{test}
	\section{test}
	\subsection{test}

	% BACK MATTER
	% ---
	% presentationback.tex

\newcommand{\commitlink}{https://github.com/CaterHatterPillar/dv2524/commit/}
\expandafter\def\expandafter\commitlink\expandafter{\commitlink \gitAbbrevHash}

\yyyymmdddate
\renewcommand{\dateseparator}{-} % To print date in same format as gitinfo-package.

\begin{frame}[t]

\begin{tabular}{p{\textwidth}}
{\small This document was compiled \today\ and last edited by \gitAuthorName\ \gitAuthorDate . This document is Version.\gitVtagn , and can be identified using revision- and commit hash \href{\commitlink}{\texttt{\gitAbbrevHash}}}.
\end{tabular}

\end{frame}

	% TODO:
	% Add colophon acknowledging progress bar.
\end{document}

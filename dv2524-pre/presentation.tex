% presentation.tex

\documentclass{beamer}
\usetheme{Intel}

\title{Paravirtualizing OpenGL ES in Simics}
\subtitle{Master's Thesis in Computer Science}
\author{Eric Nilsson}
\institute{Blekinge Institute of Technology}
\date{\today} % Fix

% Opposition presentation:
% 20 minutes
% Three key points:
% * Paravirtualization is feasible for accelerating graphics in virtual platforms.
% * Magic instructions is good methodology to transmit real-time invocations between target and host systems.
% * Paravirtualization may be subject to advanced functionality.
% The frames should lead to these conclusions.

% Screen-play:
% ---
% Front page
% Introduction:
% * Virtualization
% * Simics
% * [SHOW VIDEO]
% * Present key points
% Background:
% * Graphics virtualization
% Paravirtualization:
% * Solution architecture
% * (Man-in-the-middle windows)
% * Magic instructions and memory table traversal
% Experiment:
% * Benchmarks
% * Profiling methodology
% Results:
% * Chess results
% * Julia results
% * (Phong results)
% Conclusion:
% * Magic instruction overhead
% * Performance gains
% * Key points recap
% Acknowledgement
% Back page
% ---

\begin{document}
	% presentationfront.tex

\begin{frame}
	\titlepage
\end{frame}
	
	\begin{frame}
	\frametitle{This is the first slide}
	%Content goes here
	\end{frame}

	\begin{frame}

	\frametitle{This is the second slide}
	\framesubtitle{A bit more information about this}
	%More content goes here
	\end{frame}

	% presentationback.tex

\newcommand{\commitlink}{https://github.com/CaterHatterPillar/dv2524/commit/}
\expandafter\def\expandafter\commitlink\expandafter{\commitlink \gitAbbrevHash}

\yyyymmdddate
\renewcommand{\dateseparator}{-} % To print date in same format as gitinfo-package.

\begin{frame}[t]

\begin{tabular}{p{\textwidth}}
{\small This document was compiled \today\ and last edited by \gitAuthorName\ \gitAuthorDate . This document is Version.\gitVtagn , and can be identified using revision- and commit hash \href{\commitlink}{\texttt{\gitAbbrevHash}}}.
\end{tabular}

\end{frame}
\end{document}

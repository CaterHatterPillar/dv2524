% presentation.tex

\documentclass{beamer}
\usetheme{Intel}
%\setbeameroption{show notes} % If annotations are desired.

\usepackage{hyperref}
\hypersetup{
	colorlinks=true,
	linkcolor=blue,
	urlcolor=blue,
	citecolor=blue,
	anchorcolor=blue
}

\title{Paravirtualizing OpenGL ES in Simics}
\subtitle{Master's Thesis in Computer Science}
\author{Eric Nilsson}
\institute{Blekinge Institute of Technology}
\date{\today} % Fix

% Opposition presentation:
% 20 minutes
% Three key points:
% * Paravirtualization is feasible for accelerating graphics in virtual platforms.
% * Magic instructions is good methodology to transmit real-time invocations between target and host systems.
% * Paravirtualization may be subject to advanced functionality.
% The frames should lead to these conclusions.

% Screen-play:
% ---
% Front page
% Introduction:
% * (Virtualization)
% * Simics
% * [SHOW VIDEO]
% * Present key points
% Background:
% * Graphics virtualization
% Paravirtualization:
% * Solution architecture
% * (Man-in-the-middle windows)
% * Magic instructions and memory table traversal
% Experiment:
% * Benchmarks
% * Profiling methodology
% Results:
% * Chess results
% * Julia results
% * (Phong results)
% Conclusion:
% * Magic instruction overhead
% * Performance gains
% * Key points recap
% Acknowledgement
% Back page
% ---

\begin{document}
	% Front matter
	% presentationfront.tex

\begin{frame}
	\titlepage
\end{frame}
	
	% Body matter
	% presentationsimics.tex

\begin{frame}%[allowframebreaks]

\frametitle{Wind River\texttrademark\ Simics\texttrademark }

\begin{itemize}
	\item Full-system simulator\note{Meaning an architectural simulator which may run an unmodified software stack.}
	\item Originally devised at SICS\footnote{The Swedish Institute of Computer Science.}\note{This was the first instance of an unmodifed OS running in an entirely simulated environment.}
	\item Developed by Intel\textregistered 
	\item Sold through Intels subsidiary Wind River Systems, Inc.
	\item Used in the industry by groups such as:
	\begin{itemize}
		\item IBM
		\item NASA
		\item Lockheed Martin
	\end{itemize}
	\item Utilized extensively in academia\footnote{$300+$ universities.}
\end{itemize}

%\framebreak 

%\begin{itemize}
%	\item Deterministic Execution
%	\item Checkpointing
%	\item Reverse Execution
%\end{itemize}

\end{frame}

	% presentationsimics.tex

\begin{frame}	

\frametitle{Demonstration}

Julia Benchmark
\begin{itemize}
	\item \href{http://youtube.com/embed/GKs6OlWKFV8?rel=0&vq=hd1080&autoplay=1}{[Hardware accelerated on the simulation host]}
	\item \href{http://youtube.com/embed/3sCyzppFL0w?rel=0&vq=hd1080&autoplay=1}{[Software rasterized on the simulation target]}
	\item \href{http://youtube.com/embed/__d_EeZBzwc?rel=0&vq=hd1080&autoplay=1}{[Paravirtualized on the simulation target]}
\end{itemize}

% TODO:
% Complement with performance approximation table

\end{frame}

	% Back matter
	% presentationback.tex

\newcommand{\commitlink}{https://github.com/CaterHatterPillar/dv2524/commit/}
\expandafter\def\expandafter\commitlink\expandafter{\commitlink \gitAbbrevHash}

\yyyymmdddate
\renewcommand{\dateseparator}{-} % To print date in same format as gitinfo-package.

\begin{frame}[t]

\begin{tabular}{p{\textwidth}}
{\small This document was compiled \today\ and last edited by \gitAuthorName\ \gitAuthorDate . This document is Version.\gitVtagn , and can be identified using revision- and commit hash \href{\commitlink}{\texttt{\gitAbbrevHash}}}.
\end{tabular}

\end{frame}
\end{document}

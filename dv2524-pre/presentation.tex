% presentation.tex

\documentclass{beamer}
\usetheme{Intel}

% Document configuration
%\setbeameroption{show notes} % If annotations are desired.
\setbeamertemplate{section in toc}{\inserttocsectionnumber.~\inserttocsection} % Use number keys for table of content sections.
\setbeamertemplate{subsection in toc}{\hspace{0.5cm}\rule[0.3ex]{3pt}{3pt}~\inserttocsubsection\par} % Use a symbol for table of content subsections.
\AtBeginSection[]
{
	\begin{frame}
		\frametitle{Table of Contents}
		\begin{multicols}{2}
			\tableofcontents[currentsection]
		\end{multicols}
    \end{frame}
} % Insert table of contents frame before each new section.
\setbeamertemplate{caption}[numbered] % Number figures.

% Package inclusion:
\usepackage{tikz} % Used by presentationprogressbar.tex
\usepackage{hyperref} % Used to invoke Santa Claus.
\usepackage{multicol} % Used to split up table of contents in a double column style.
\usepackage{multirow} % Needed for multirow graphs
\usepackage{datetime} % Used to format dates.
\usepackage{pdflscape}

\usepackage[footinfo]{gitinfo} % dv2524-pac/

% Package configuration:
\hypersetup{
	colorlinks=true,
	linkcolor=blue,
	urlcolor=black,
	citecolor=blue,
	anchorcolor=blue
}
\usetikzlibrary{calc}

% presentationprogressbar.tex
% Defines a progress bar used at the top of each frame.
% ---
% Written by Gonzalo Medina (http://tex.stackexchange.com/users/3954/gonzalo-medina).
% http://tex.stackexchange.com/questions/59742/progress-bar-for-latex-beamer
% Jun 13 '12
% edited Jun 21 '12

\definecolor{pbblue}{HTML}{0A75A8}% color for the progress bar and the circle 

\makeatletter
\def\progressbar@progressbar{} % the progress bar
\newcount\progressbar@tmpcounta% auxiliary counter
\newcount\progressbar@tmpcountb% auxiliary counter
\newdimen\progressbar@pbht %progressbar height
\newdimen\progressbar@pbwd %progressbar width
\newdimen\progressbar@rcircle % radius for the circle
\newdimen\progressbar@tmpdim % auxiliary dimension

\progressbar@pbwd=\linewidth
\progressbar@pbht=1pt
\progressbar@rcircle=2.5pt

% the progress bar
\def\progressbar@progressbar{%

    \progressbar@tmpcounta=\insertframenumber
    \progressbar@tmpcountb=\inserttotalframenumber
    \progressbar@tmpdim=\progressbar@pbwd
    \multiply\progressbar@tmpdim by \progressbar@tmpcounta
    \divide\progressbar@tmpdim by \progressbar@tmpcountb

  \begin{tikzpicture}
    \draw[pbblue!30,line width=\progressbar@pbht]
      (0pt, 0pt) -- ++ (\progressbar@pbwd,0pt);

    \filldraw[pbblue!30] %
      (\the\dimexpr\progressbar@tmpdim-\progressbar@rcircle\relax, .5\progressbar@pbht) circle (\progressbar@rcircle);

    \node[draw=pbblue!30,text width=3.5em,align=center,inner sep=1pt,
      text=pbblue!70,anchor=east] at (0,0) {\insertframenumber/\inserttotalframenumber};
  \end{tikzpicture}%
}

\addtobeamertemplate{headline}{}
{%
  \begin{beamercolorbox}[wd=\paperwidth,ht=4ex,center,dp=1ex]{white}%
    \progressbar@progressbar%
  \end{beamercolorbox}%
}
\makeatother

% Command to read, and ouput the first line of a file:
\newread\file
\newcommand{\dvtcmdfirstline}[1]{
	\openin\file=#1
	\read\file to \keyval
	\keyval % Return keyval.
	\closein\file
}

\title{Paravirtualizing OpenGL ES in Simics}
\subtitle{Master's Thesis in Computer Science}
\author{Eric Nilsson}
\institute{Blekinge Institute of Technology}
\date{\today} % Fix

% Opposition presentation:
% 20 minutes
% Three key points:
% * Paravirtualization is feasible for accelerating graphics in virtual platforms.
% * Magic instructions is good methodology to transmit real-time invocations between target and host systems.
% * Paravirtualization may be subject to advanced functionality.
% The frames should lead to these conclusions.

\newtheorem{thm}{Key point}

\begin{document}
	% FRONT MATTER
	% ---
	% presentationfront.tex

\begin{frame}
	\titlepage
\end{frame}
	
	% BODY MATTER
	% ---
	% INTRODUCTION
	\section{Introduction}
	% Simics
	\subsection{Simics}
	% presentationsimics.tex

\begin{frame}%[allowframebreaks]

\frametitle{Wind River\texttrademark\ Simics\texttrademark }

\begin{itemize}
	\item Full-system simulator\note{Meaning an architectural simulator which may run an unmodified software stack.}
	\item Originally devised at SICS\footnote{The Swedish Institute of Computer Science.}\note{This was the first instance of an unmodifed OS running in an entirely simulated environment.}
	\item Developed by Intel\textregistered 
	\item Sold through Intels subsidiary Wind River Systems, Inc.
	\item Used in the industry by groups such as:
	\begin{itemize}
		\item IBM
		\item NASA
		\item Lockheed Martin
	\end{itemize}
	\item Utilized extensively in academia\footnote{$300+$ universities.}
\end{itemize}

%\framebreak 

%\begin{itemize}
%	\item Deterministic Execution
%	\item Checkpointing
%	\item Reverse Execution
%\end{itemize}

\end{frame}

	% Demonstration
	\subsection{Demonstration}
	% presentationsimics.tex

\begin{frame}	

\frametitle{Demonstration}

Julia Benchmark
\begin{itemize}
	\item \href{http://youtube.com/embed/GKs6OlWKFV8?rel=0&vq=hd1080&autoplay=1}{[Hardware accelerated on the simulation host]}
	\item \href{http://youtube.com/embed/3sCyzppFL0w?rel=0&vq=hd1080&autoplay=1}{[Software rasterized on the simulation target]}
	\item \href{http://youtube.com/embed/__d_EeZBzwc?rel=0&vq=hd1080&autoplay=1}{[Paravirtualized on the simulation target]}
\end{itemize}

% TODO:
% Complement with performance approximation table

\end{frame}
	% Key points
	\subsection{Key points}
	% presentationkeypoints.tex

\begin{frame}
\frametitle{Key points}

\begin{thm}<1->
	Paravirtualization is feasible for accelerating graphics in virtual platforms.
\end{thm}

\begin{thm}<1->
	Magic instructions is promising methodology to carry real-time invocations between target and host systems.
\end{thm}

\begin{thm}<1->
	Paravirtualization may be subject to deterministic execution, checkpointing, and reverse execution.
\end{thm}

\end{frame}


	% BACKGROUND
	\section{Background}
	% Graphics virtualization
	\subsection{Graphics virtualization}
	% presentationgraphicsvirtualization.tex

\begin{frame}
\frametitle{Graphics virtualization}

\begin{columns}
	\column{0.5\textwidth}
	\begin{block}{GPU modeling}
		Develop a GPU model virtualizing the GPU Instruction Set Architecture (ISA).
	\end{block}
	\begin{block}{PCI passthrough}
		Utilize passthrough methodology; granting virtual systems first-hand access to host machine devices.
	\end{block}
    \column{0.5\textwidth}
    \begin{block}{Soft modeling}
    	Use advanced software rasterizers optimizing GPU kernel simulation for CPU architectures.
    \end{block}
    \begin{block}{Paravirtualization}
    	Selectively modify the virtual architecture to accomodate scalability, performance, and simplicity.
    \end{block}
\end{columns}
	
\end{frame}

	% PARAVIRTUALIZATION
	\section{Paravirtualization}
	\subsection{Overview}
	% presentationoverview.tex

\begin{frame}
\frametitle{Overview}

\begin{center}
	\includegraphics[height=0.8\textheight]{yedoverview.pdf}
\end{center}

\end{frame}

	% Man-in-the-middle windows
	\subsection{Man-in-the-middle windows}
	% presentationwindows.tex

\begin{frame}
\frametitle{Man-in-the-middle windows}

\ldots

\end{frame}
	% Magic instructions
	\subsection{Magic instructions}
	% presentationmagicinstructions.tex

\begin{frame}
\frametitle{Magic instructions}

\begin{block}{Background}
	\begin{itemize}
		\item Many resons to escape simulation
		\begin{itemize}
			\item Breakpoints
			\item File sharing
			\item Profiling
		\end{itemize}
		\item Several methodologies
	\end{itemize}
\end{block}

\begin{block}{The 'Magic instruction'}
	\begin{itemize}
	\item \dvtcmdcodeinline{nop}-type instruction\footnote{\dvtcmdcodeinline{xchg ebx, ebx}}
	\item Potentially instant
	\item No modification to target system
\end{itemize}
\end{block}

\end{frame}
 % W. memory table traversal

	% EXPERIMENT
	\section{Experiment}
	% Benchmarks
	\subsection{Benchmarks}
	% presemtationbenchmarks.tex

\begin{frame}
\frametitle{Benchmarks}

% figbenchmarks.tex

\begin{figure}

\minipage{0.32\textwidth}
	\includegraphics[width=\linewidth]{imgchess.png}
	\caption{Chess.}
	\label{fig:benchmarks_chess}
\endminipage\hfill
\minipage{0.32\textwidth}
	\includegraphics[width=\linewidth]{imgjulia.png}
	\caption{Julia.}
	\label{fig:benchmarks_julia}
\endminipage\hfill
\minipage{0.32\textwidth}
  \includegraphics[width=\linewidth]{imgphong.png}
  \caption{Phong.}
  \label{fig:benchmarks_phong}
\endminipage

\end{figure}
	
\end{frame}
	% Profiling
	\subsection{Profiling}
	% presentationprofiling.tex

\begin{frame}
\frametitle{Profiling}

FIGURE
	
\end{frame}

	% RESULTS
	\section{Results}
	% Results - Host
	\subsection{Reference histograms}
	% presentationresultshost.tex

\begin{frame}
\frametitle{Results - Reference}

\begin{columns}
	\column{0.5\textwidth}
	\centering
	% histogramshost
	\resizebox{0.8\textwidth}{!}{%
		\input{gnuhistogramshost.tex}
	}
    \column{0.5\textwidth}
    \centering
    % histogramsqemu
	\resizebox{0.8\textwidth}{!}{%
		\input{gnuhistogramsqemu.tex}
	}
\end{columns}
	
\end{frame}
	% Results - Chess
	\subsection{Chess histograms}
	% presentationresultschess.tex

\begin{frame}
\frametitle{Chess histograms}

\begin{center}
\resizebox{0.8\textwidth}{!}{%
	\input{gnuhistogramssimicsparachess.tex}
}
\end{center}

\end{frame}
	% Results - Julia
	\subsection{Julia histograms}
	% presentationresultsjulia.tex

\begin{frame}
\frametitle{Simics histograms: Julia}

\begin{center}
\resizebox{0.8\textwidth}{!}{%
	\input{gnuhistogramssimicsparajulia.tex}
}
\end{center}

\end{frame}
	% Results - Phong
	\subsection{Phong histograms}
	% presentationresultsphong.tex

\begin{frame}
\frametitle{Results - Phong}

\begin{center}
\resizebox{0.8\textwidth}{!}{%
	\input{gnuhistogramssimicsparaphong.tex}
}
\end{center}

\end{frame}

	\subsection{Tables}
	% presentationtables.tex

\begin{landscape}
\begin{frame}
\frametitle{}
\begin{center}

Software rasterized:
% keyvalsimics
\resizebox{\linewidth}{!}{%
	% tabkeyvalsimics.tex
% Defines a table outlining the performance of the thesis benchmark whilst hardware accelerated on the host platform.

% This ought be made into an environment.

\begin{table}
\begin{center}

\newcommand{\chesskeyone}{60x60 tiles}
\newcommand{\chesskeytwo}{84x84 tiles}
\newcommand{\chesskeythree}{118x118 tiles}

\newcommand{\juliakeyone}{225 iterations}
\newcommand{\juliakeytwo}{450 iterations}
\newcommand{\juliakeythree}{900 iterations}

\newcommand{\phongkeyone}{1448x1448 texels}
\newcommand{\phongkeytwo}{2048x2048 texels}
\newcommand{\phongkeythree}{2896x2896 texels}

\begin{tabular}{|c|c|c|c|c|c|}
\hline
\multirow{2}{*}{Benchmark} & \multirow{2}{*}{Key} & \multicolumn{4}{p{6cm}|}{\centering Elapsed time in Milliseconds} \\
\cline{3-6} && \multicolumn{1}{c|}{Min} & \multicolumn{1}{c|}{Max} & \multicolumn{1}{c|}{Std} & \multicolumn{1}{c|}{Avg} \\ \hline
\multirow{3}{*}{Chess} & \chesskeyone & \dvtcmdfirstline{simicschess60x60.dat.min}		& \dvtcmdfirstline{simicschess60x60.dat.max}		& \dvtcmdfirstline{simicschess60x60.dat.std}		& \dvtcmdfirstline{simicschess60x60.dat.avg} \\ %\cline{2-6}
& \chesskeytwo & \dvtcmdfirstline{simicschess84x84.dat.min} & \dvtcmdfirstline{simicschess84x84.dat.max} & \dvtcmdfirstline{simicschess84x84.dat.std} & \dvtcmdfirstline{simicschess84x84.dat.avg} \\ %\cline{2-6}
& \chesskeythree & \dvtcmdfirstline{simicschess118x118.dat.min} & \dvtcmdfirstline{simicschess118x118.dat.max} & \dvtcmdfirstline{simicschess118x118.dat.std} & \dvtcmdfirstline{simicschess118x118.dat.avg} \\ \hline
\multirow{3}{*}{Julia} & \juliakeyone & \dvtcmdfirstline{simicsjulia225.dat.min}		& \dvtcmdfirstline{simicsjulia225.dat.max}		& \dvtcmdfirstline{simicsjulia225.dat.std} & \dvtcmdfirstline{simicsjulia225.dat.avg} \\ %\cline{2-6}
& \juliakeytwo & \dvtcmdfirstline{simicsjulia450.dat.min} & \dvtcmdfirstline{simicsjulia450.dat.max} & \dvtcmdfirstline{simicsjulia450.dat.std} & \dvtcmdfirstline{simicsjulia450.dat.avg} \\ %\cline{2-6}
& \juliakeythree & \dvtcmdfirstline{simicsjulia900.dat.min} & \dvtcmdfirstline{simicsjulia900.dat.max} & \dvtcmdfirstline{simicsjulia900.dat.std} & \dvtcmdfirstline{simicsjulia900.dat.avg} \\ \hline
\multirow{3}{*}{Phong} & \phongkeyone & \dvtcmdfirstline{simicsphong1448x1448.dat.min}		& \dvtcmdfirstline{simicsphong1448x1448.dat.max}		& \dvtcmdfirstline{simicsphong1448x1448.dat.std}		& \dvtcmdfirstline{simicsphong1448x1448.dat.avg} \\ %\cline{2-6}
& \phongkeytwo & \dvtcmdfirstline{simicsphong2048x2048.dat.min} & \dvtcmdfirstline{simicsphong2048x2048.dat.max} & \dvtcmdfirstline{simicsphong2048x2048.dat.std} & \dvtcmdfirstline{simicsphong2048x2048.dat.avg} \\ %\cline{2-6}
& \phongkeythree & \dvtcmdfirstline{simicsphong2896x2896.dat.min} & \dvtcmdfirstline{simicsphong2896x2896.dat.max} & \dvtcmdfirstline{simicsphong2896x2896.dat.std} & \dvtcmdfirstline{simicsphong2896x2896.dat.avg} \\ \hline
\end{tabular}

\label{tab:tabletest}
\caption{A test table}

\end{center}
\end{table}
}

\vspace{1cm}

Paravirtualized:
% keyvalpara
\resizebox{\linewidth}{!}{%
	% tabkeyvalpara.tex
% Defines a table outlining the performance of the thesis benchmark whilst paravirtualized in the para full-system simulator.

% This ought be made into an environment.

\begin{table}
\begin{center}

\newcommand{\chesskeyone}{60x60 tiles}
\newcommand{\chesskeytwo}{84x84 tiles}
\newcommand{\chesskeythree}{118x118 tiles}

\newcommand{\juliakeyone}{225 iterations}
\newcommand{\juliakeytwo}{450 iterations}
\newcommand{\juliakeythree}{900 iterations}

\newcommand{\phongkeyone}{1448x1448 texels}
\newcommand{\phongkeytwo}{2048x2048 texels}
\newcommand{\phongkeythree}{2896x2896 texels}

\begin{tabular}{|c|c|c|c|c|c|}
\hline
\multirow{2}{*}{Benchmark} & \multirow{2}{*}{Key} & \multicolumn{4}{p{6cm}|}{\centering Elapsed time (milliseconds)} \\
\cline{3-6} && \multicolumn{1}{c|}{Min} & \multicolumn{1}{c|}{Max} & \multicolumn{1}{c|}{Std} & \multicolumn{1}{c|}{Avg} \\ \hline
\multirow{3}{*}{Chess} & \chesskeyone & \dvtcmdfirstline{parachess60x60.dat.min}		& \dvtcmdfirstline{parachess60x60.dat.max}		& \dvtcmdfirstline{parachess60x60.dat.std}		& \dvtcmdfirstline{parachess60x60.dat.avg} \\ %\cline{2-6}
& \chesskeytwo & \dvtcmdfirstline{parachess84x84.dat.min} & \dvtcmdfirstline{parachess84x84.dat.max} & \dvtcmdfirstline{parachess84x84.dat.std} & \dvtcmdfirstline{parachess84x84.dat.avg} \\ %\cline{2-6}
& \chesskeythree & \dvtcmdfirstline{parachess118x118.dat.min} & \dvtcmdfirstline{parachess118x118.dat.max} & \dvtcmdfirstline{parachess118x118.dat.std} & \dvtcmdfirstline{parachess118x118.dat.avg} \\ \hline
\multirow{3}{*}{Julia} & \juliakeyone & \dvtcmdfirstline{parajulia225.dat.min}		& \dvtcmdfirstline{parajulia225.dat.max}		& \dvtcmdfirstline{parajulia225.dat.std} & \dvtcmdfirstline{parajulia225.dat.avg} \\ %\cline{2-6}
& \juliakeytwo & \dvtcmdfirstline{parajulia450.dat.min} & \dvtcmdfirstline{parajulia450.dat.max} & \dvtcmdfirstline{parajulia450.dat.std} & \dvtcmdfirstline{parajulia450.dat.avg} \\ %\cline{2-6}
& \juliakeythree & \dvtcmdfirstline{parajulia900.dat.min} & \dvtcmdfirstline{parajulia900.dat.max} & \dvtcmdfirstline{parajulia900.dat.std} & \dvtcmdfirstline{parajulia900.dat.avg} \\ \hline
\multirow{3}{*}{Phong} & \phongkeyone & \dvtcmdfirstline{paraphong1448x1448.dat.min}		& \dvtcmdfirstline{paraphong1448x1448.dat.max}		& \dvtcmdfirstline{paraphong1448x1448.dat.std}		& \dvtcmdfirstline{paraphong1448x1448.dat.avg} \\ %\cline{2-6}
& \phongkeytwo & \dvtcmdfirstline{paraphong2048x2048.dat.min} & \dvtcmdfirstline{paraphong2048x2048.dat.max} & \dvtcmdfirstline{paraphong2048x2048.dat.std} & \dvtcmdfirstline{paraphong2048x2048.dat.avg} \\ %\cline{2-6}
& \phongkeythree & \dvtcmdfirstline{paraphong2896x2896.dat.min} & \dvtcmdfirstline{paraphong2896x2896.dat.max} & \dvtcmdfirstline{paraphong2896x2896.dat.std} & \dvtcmdfirstline{paraphong2896x2896.dat.avg} \\ \hline
\end{tabular}

\label{tab:keyvalpara}
\caption{Benchmarking results whilst paravirtualized in the Simics full-system simulator.}

\end{center}
\end{table}
}

\end{center}
\end{frame}
\end{landscape}


	% CONCLUSION
	\section{Conclusion}
	% Performance improvements
	\subsection{Performance improvements}
	% presentationperformance.tex

\begin{frame}
\frametitle{Performance improvements}

\ldots

\end{frame}
	% Magic instruction overhead
	\subsection{Magic instruction overhead}
	% presentationmagicinstructionoverhead.tex

\begin{frame}[fragile]
\frametitle{Magic instruction overhead}
\begin{figure}
\centering
\begin{lstlisting}
int sercon = open("/dev/ttyS0",O_RDWR|O_NOCTTY);
\end{lstlisting}
\begin{minipage}{.5\textwidth}
	\centering
	
\lstinputlisting{pseudoindividual.txt}
\end{minipage}%
\begin{minipage}{.5\textwidth}
	\centering

\lstinputlisting{pseudobatch.txt}
\end{minipage}
\end{figure}

\begin{center}
% tabmagicinstructionsforall.tex

\begin{tabular}{llll}
Min & Max & Std & Avg \\ \hline
\dvtcmdfirstline{magicinstrprofileall.dat.min} & \dvtcmdfirstline{magicinstrprofileall.dat.max} & \dvtcmdfirstline{magicinstrprofileall.dat.std} & \dvtcmdfirstline{magicinstrprofileall.dat.avg} \\
\end{tabular}

\end{center}

\end{frame}

	% Advanced functionality
	\subsection{Advanced functionality}
	% presentationadvancedfunctionality.tex

\begin{frame}
\frametitle{Advanced functionality}

\begin{itemize}
	\item Deterministic execution
	\item Checkpointing
	\item Reverse execution
\end{itemize}

\end{frame}
	% Key point recap
	\subsection{Key point recap}
	% presentationkeypoints.tex

\begin{frame}
\frametitle{Key points}

\begin{thm}<1->
	Paravirtualization is feasible for accelerating graphics in virtual platforms.
\end{thm}

\begin{thm}<1->
	Magic instructions is promising methodology to carry real-time invocations between target and host systems.
\end{thm}

\begin{thm}<1->
	Paravirtualization may be subject to deterministic execution, checkpointing, and reverse execution.
\end{thm}

\end{frame}


	% BACK MATTER
	% ---
	% presentationback.tex

\newcommand{\commitlink}{https://github.com/CaterHatterPillar/dv2524/commit/}
\expandafter\def\expandafter\commitlink\expandafter{\commitlink \gitAbbrevHash}

\yyyymmdddate
\renewcommand{\dateseparator}{-} % To print date in same format as gitinfo-package.

\begin{frame}[t]

\begin{tabular}{p{\textwidth}}
{\small This document was compiled \today\ and last edited by \gitAuthorName\ \gitAuthorDate . This document is Version.\gitVtagn , and can be identified using revision- and commit hash \href{\commitlink}{\texttt{\gitAbbrevHash}}}.
\end{tabular}

\end{frame}

	% TODO:
	% Add colophon acknowledging progress bar.
\end{document}

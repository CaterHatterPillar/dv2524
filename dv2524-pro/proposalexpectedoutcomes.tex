\chapter{Expected~Outcomes}
\label{cha:expectedoutcomes}
Paravirtualization may be defined as the notion of modifying a \termtarget\ \termos\ in order to circumvent functionality that may otherwize obstruct efficient \termisa\ simulation\dvpcmdcitebib[p.~422]{publications:smith:2005}.
Using this methodology, one may substantially increase the performance of some virtualized systems (\dvcmdrefsee{\dvpcmdcitebib[p.~422]{publications:smith:2005}} for an example).
By implementing a paravirtualized solution for graphics acceleration in \termsimics , we hope to achieve an increased understanding of paravirtualized drivers in virtual platforms. Additionally, proposed study ought to strive to identify difficulties and challenges that obstruct paravirtualization in system simulators. As such, an expected outcome of proposed study is a technical report extensive enough to aid those wishing to paravirtualize graphics or other drivers in system simulators.

Additionally, we aim to analyze the attributes of paravirtualized graphics from the perspective of what is considered desireable from a full-system simulator such as \termsimics .
This stresses the importance of determinism and repeatability, and involves speculation into what obstacles may follow paravirtualized \termopengles .
A speculative example of such an issue is presented in \dvcmdrefname{cha:appendixa} (\dvcmdrefappa{sec:appendixa_determinisminopengles}).\\

\noindent
Throughout the pilot study performed for the purpose of the study proposed in this document, there has been no indication in academic writing of any pre-existing solution of paravirtualized graphics \termapi s signifying deterministic behaviour with the ability to store and restore the state of a simulation.
Such functionality could simplify debugging, testing, and profiling of applications comprising some \termgpu -bound workload; not limited to graphics or \termgpu\ utilization in its entirety.
As such, proposed study is relevant for the field of \termcompsci\ in terms of facilitating debugging, testing, and profiling of software dependent on \termgpu s, whilst maintaining feasible performance throughout simulation.

%We expect a performance improvement signifying feasability of real-time graphics, i.e. 30 - or exceeding 30 - \termfps\ for some benchmark, in \termsimics . Additionally, 

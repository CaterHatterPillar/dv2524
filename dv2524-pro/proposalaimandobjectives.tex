% proposalaimandobjectives.tex

\chapter{Aim~\&~Objectives}
\label{cha:aimandobjectives}
The scope of the study proposed in this document consist of implementing paravirtualization of a graphics \termapi\ (being \termopengles ~2.0) in the \termsimics\ full system simulator developed by \termintel\ and sold through \termintel 's subsidiary \termwindriver\
Said implementation adheres to a certain \termrefimpl\ (\dvcmdrefsee{\dvcmdrefsec{sec:researchmethodology_referencesolution}}), which is expanded upon in the latter parts of this document.
As such, the aim concerns investigating the performance, and the feasibility of extended benefits \& advanced functionality, of paravirtualized graphics acceleration in a virtual platform.
Said integration entails investigating, analyzing, and developing methods \& techniques for efficient communication and execution in the \termsimics\ run-time environment, in addition to identifying liabilities of the paravirtualized solution in regards to \termsimics\ philosophy (being high-performance and \termtiming -accurate determinism and repeatability\dvpcmdcitebib{journals:aarno:2013}).
As such, proposed study does not exclusively concern \termsimics\ integration, but an investigation into paravirtualized drivers in virtual platforms.
The key aspects of such integration are as follows:
\begin{enumerate*}
	\item \label{itm:enum_aspects_performance} Performance characteristics.
	Comparing the performance of a paravirtualized solution to that of a software renderer (\termsimics\ vs. \termsimics ).
	Comparing the \termsimics\ solution to the \termrefsolu\ to see if the implementation carried over the same benefits (\termsimics\ vs. \termrefsolu ).
	\item \label{itm:enum_aspects_advancedanalyze} A naïve porting of the \termrefimpl 's \termopengles\ acceleration into \termsimics\ would not support advanced features such as \termdetexe , \termcheckpointing , or \termrevexe .
	Therefore, it would be beneficial to analyze the solution in terms of what it would take to support these features.
	\item \label{itm:enum_aspects_advancedimplement} Extend the \termopengles\ acceleration to support \termcheckpointing\ and \termrevexe .
\end{enumerate*}
The proposed scope for said study is thus considered to investigate (\ref{itm:enum_aspects_performance}) and (\ref{itm:enum_aspects_advancedanalyze}) with (\ref{itm:enum_aspects_performance}) being the focal point of the study and (\ref{itm:enum_aspects_advancedimplement}), or part of (\ref{itm:enum_aspects_advancedimplement}), being the stretch goal.
As such, the objectives that follow from the stated aim may be summarized as follows (\dvcmdrefsee{\dvcmdrefcha{cha:researchmethodology}} for more implementational details):

\newcommand*\objective[1]{\item}%\item[\alph]
\begin{multicols}{2}
\begin{enumerate}[(a)]
	\objective{1} Establish performance-wize sound communications between \termhost\ and \termtarget .
	\objective{2} Be able to send, receive, and callback, data in-between \termhost\ and \termtarget .
	\objective{3} Translate given methods into desktop \termapi\ calls and invoke translated methods on the \termhost\ system.
	\objective{4} Display results in \termtarget\ system.
	\objective{5} Analyze performance variations of paravirtualized solution in regard to existing software~rasterizer in \termsimics , as described in (\ref{itm:enum_aspects_performance}).
	\objective{6} Investigate feasibility of advanced functionality described in (\ref{itm:enum_aspects_advancedanalyze}).
\end{enumerate}
\end{multicols}

\section{Industrial~Collaboration}
\label{sec:aimandobjectives_industrialcollaboration}
Proposed study is performed at \termbth , for the \termbthdept , in collaboration with \termintel .
In this manner, access to the source code of \termsimics\ along with documentation and other resources surrounding the \termsimics\ environment will be available during proposed study.
During this time, the author will be employed at \termintel\ and will execute and compile the thesis at the \termintel\ offices in Stockholm.

\section{Related~Work}
\label{sec:aimandobjectives_relatedwork}
System simulators are abundant and exist in corporate\dvpcmdcitebib{magazines:bohrer:2004}, academic\dvpcmdcitebib{journals:rosenblum:1995}, and open-source variations\dvpcmdciteref{magazines:bartholomew:2006}.
These platforms, such as \termsimics , have been used for a variety of purposes including, but not limited to, thermal control strategies in multicores\dvpcmdcitebib{inproceedings:bartolini:2010}, networking timing analysis\dvpcmdcitebib{journals:ortiz:2009}, web server performance evaluation\dvpcmdcitebib{journals:villa:2005}, and to simulate costly hardware financially unfeasible to researchers\dvpcmdcitebib{journals:alameldeen:2003}.
A wide array of corporations employ \termsimics , including \termibm \dvpcmdcitebib[p.~12:1,~12:6]{journals:koerner:2009}, \termintel \dvpcmdcitebib[p.~100]{journals:veselyi:2013}, and \termnasa \dvpcmdciteref{web:windriver:2014}\dvpcmdciteref{web:nasa:2014}.
Additionally, the simulator has been known to operate in over $300$ universities throughout the world\dvpcmdcitebib[p.~252]{journals:villa:2005}.\\

\noindent
\termqemu \footnote{'Quick~Emulator'.} is an open-source virtual platform described as a full system emulator\dvpcmdcitebib[p.~1]{inproceedings:bellard:2005} (\dvcmdrefsee{\dvpcmdciteref[p.~69]{magazines:bartholomew:2006}} for an overview of \termqemu ) and a high-speed functional simulator\dvpcmdcitebib[p.~1]{inproceedings:shen:2010}.
Its virtual platform supports simulation of several common system architectures and hardware devices and can, like \termsimics , save and restore the state of a simulation\dvpcmdcitebib[p.~1]{inproceedings:bellard:2005}.
\termqemu\ is an extensively used virtual platform, and is the simulator used in the \termrefsolu\ (\dvcmdrefsee{\dvcmdrefsec{sec:researchmethodology_referencesolution}}).\\

\noindent
The concepts of advanced simulatory features are not new to \termgpu\ virtualization, as there have been previous attempts to implement the \termcheckpointrestart -model in a \termgpu\ context\dvpcmdcitebib{inproceedings:guo:2013}.
Similar works include modeling \termgpu\ devices in \termqemu\ with software \termopengles\ rasterization support\dvpcmdcitebib{inproceedings:shen:2010}.

% Stuff to look up:
% * Any more available info on Virgil3D?
% * Previous work on GPU Reverse Execution.
% * Previous work on Determinism in GPU technologies.
% * Previous work on GPU simulation, in general.

% Unsupported claims:
%Albeit supporting \termgpu\ virtualization, at version $1.7$ \termqemu\ still, in and of its own, lack support for hardware accelerated rasterization\cite{NEEDED}.
%However, as early as 2006 there were experimental attempts at accelerating \termopengl\ in \termqemu \cite{SEE}.
%The \termqemu\ platform has since been the focal point of several attempts at achieving such acceleration, possibly due to its open-source licensing.
%Such an attempt is the \termvirgilthreed\ virtual \termgpu-project\cite{SEE}, that strives to create an abstract \termtarget\ \termgpu\ which may utilize the \termhost\ device.

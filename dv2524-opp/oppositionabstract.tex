% oppositionabstract.tex

\abstract
\begin{changemargin}{+1cm}{+1cm}
\noindent
\textbf{Opposition}.
This report presents an opposition to a Master Thesis accomplished at Blekinge Institute of Science in partial fulfilment for the degree of Master of Science in Computer Science.
The opposition consists of a summary of the subject in question, along with a brief analysis on the validity of the findings presented, their contribution to the field, and a review of the thesis; in terms of benefits and drawbacks in its presentation.
Enclosed is a draft copy of the thesis in question, along with a scanned copy of a hand-written annotation of said thesis.
This opposition is authored in goodwill by Nilsson E; friend and classmate of the authors of the opposed dissertation.

The contents of this document adheres to the thesis opposition guidelines as presented by the Royal Institute of Technology under the title \textit{OPPOSITION FOR MASTER'S PROJECT} (SE: \textit{EXJOBBSOPPOSITION}).
\newline
\textbf{Thesis}. The tentative title of the opposed thesis presented in this document is \textit{Software defect prediction using machine learning on test and source code metrics}.
The dissertation is written by Liljeson M. and Mohlin. A., and is a thesis draft submitted to the Blekinge Institute of Technology for the degree of Master of Science in Computer Science.
\newline
\textbf{Summary}.
The presented subject concerns inquiry into software defect prediction with the purpose of analyzing whether or not a combination of metrics from source code and corresponding tests may incur more accurate fault prediction.
The thesis is comprised of a presentation of a literature study performed for the sake of the study, along with a subsequent description on experiment methodology, an analysis of the produced results, and recommendations for future work in the area.
\newline
\textbf{Conclusion}.
The opposition concludes that the  opposed thesis, authored by Liljeson M. and Mohlin A., constitutes a contribution to the field of metrics and models in software engineering, and that the thesis is fit for publishing.
Furthermore, the opposition features a number of suggested improvements consituting grammar, formatting, visualization, suggested elaboration, and further conclusions.

\par\vspace {0.5cm}
\noindent
\textbf{Keywords:} software defect prediction; software testing; machine learning;

\end{changemargin}

% oppositionpresentation.tex

\paragraph{Opposition: Presentation}
\label{par:oppositionpresentation}

\subparagraph{Opposition: Formatting}
\label{par:oppositionformatting}
In general, the material presented in the concerned thesis is well-formatted with correct paragraph structure and character returns.
The thesis is well-written and easy to read.
It is apperent that the authors have put energy into formatting of the document and its presentation
The phrasing is not superlatively expressed and section- and figure cross-referencing is sound.

Citations are correctly formatted, but inconsistent in their inclusion of page numbers.
As such, it may be viable to decide on one standard (either featuring page numbers entirely or not at all), unless there is a reason as to why the authors break the trend.

Furthermore, abbreviations and terminology are generally introduced and explained in a correct manner - most often introducing concepts before they are discussed as should be.

The negatives in the thesis presentation are comprised of the occational parenthesis-enclosed text that may be better formatted as a footnote (see attached annotated thesis).
Additionally, the author would like to express his concern with percieved over-utilization of the \LaTeX\ \texttt{subsubsection} environment, which renders the material structure less comprehensible.
It may be viable to consider instead using the \texttt{paragraph} environment, or simply not segment the material into so many sections.
However, this may be a question on personal preference.

\subparagraph{Opposition: Delivery}
\label{par:oppositionpresentation_delivery}
Thesis presentation may be greatly improved by the addition of an acronym list (due to the widespread abbreviational use) and a terminology (due to the use of many different, often obscurely named, algorithms and methods).
Said lists could serve as a legend to the reader, when reading the thesis, as the wide array of concepts, algorithms, and abbreviations may otherwise obstruct the understanding of the source material.

The title is concise and briefly portrays the contents of the thesis.
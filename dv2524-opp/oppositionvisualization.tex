% oppositionvisualization.tex

\paragraph{Opposition: Visualization}
\label{par:oppositionvisualization}
Opposed thesis features a number of well-design figures and graphs; figures apparent that the authors have put energy into shaping.
However, the author would like to describe a number of ways in which the visualization of the thesis could be improved.
These suggestions, on a per-figure basis, are presented below.

\subparagraph{Opposition: Figure 3.2:}
\label{par:oppositionvisualization_figurethreepointtwo}
The author percieves it as if the introductory text under section Metric sets describes the relationship in-ebtween trouble reports and metric extraction better than the enclosed figure.
As such, one ought reconsider what the visualization adds to the material.
If said visualization is redundant, one might consider removing it from the document.

\subparagraph{Opposition: Figure 4.4}
\label{par:oppositionvisualization_figurefourpointfour}
As far as the author can tell, there are no references to specific lines for the pseudocode presented in figure 4.4.
If so is the case, line numbers simply clutter the pseudocode.

\subparagraph{Opposition: Figures 5.1-5.4}
\label{par:oppositionvisualization_figuresfivepointonetofivepointfour}
If the thesis is intended to be read on printed paper, rather than on an electronical device using which one may zoom-in on figures, the author would like to suggest not featuring a percantage Y-axis covering the entirety of the 0-100\% spectrum.
Rather, for the purposes of visualization, it may be benificial to only present data from the lowest percentage point and up; considering the large whitespace covering some of the plots.

Assuming the preposition as outlined above, the graphs also feature outliers too small - especially considering printers with lower DPI-count.
As such, one should consider the necessity of said outliers; and if they contribute in a substantial way to the data being presented.

Furthermore, for the sake of reference overview, it may be benificial to feature more detailed captions so that the reader may relate to the graphs without having read the corresponding text section beforehand.

\subparagraph{Opposition: Figures 5.5-5.7}
\label{par:oppositionvisualization_figuresfivepointfivetofivepointseven}
These figures features a large amount of whitespace, like Figures 5.1-5.4, which may be reduced by the visualization not covering the entirety of the span of the Y-axis.
Albeit not detrimental to the visualization due to the large bars, reducing said whitespace may simply make the figure more compact.

\subparagraph{Opposition: Figures 5.6-5.9}
\label{par:oppositionvisualization_figuresfivepointsixtofivepointnine}
The author would like to commend the original authors on a great visualization fit-for-the-purpose.
Albeit the colors are not hinders to viewing the visualization in grayscale, the suggestions presented in paragraph \nameref{par:oppositionvisualization_colors} may still aid in said visualization.

\subparagraph{Opposition: Colors}
\label{par:oppositionvisualization_colors}
Several of the figures featured in the opposed thesis are unsuitable for grayscale printing (see \nameref{par:oppositionvisualization_figuresfivepointonetofivepointfour}).
In the authors personal preference, papers ought be written - to the best of ones ability - to accomodate grayscale printing.
However, the author wishes to express whether or not one regards grayscale visualization adaption to be entirely up to personal preference; making it a less relevant criticism.
The reason the author yet expresses this concern is due to possible trivial changes to enhance grayscale adaptiveness.

As such, the author wishes to suggest using \textit{box fills} to, instead of coloring the box-and-whiskers plots, fill the boxes using dotted, striped, empty, and full box styles.
This would render the graphs more easily interpreted when printed in grayscale.

However, and again, this may be entirely up to personal preference.
If the original authors of the document advise the reader to examine their thesis in color-print, I see no reason as to why the suggestions in this paragraph might be considered valid criticism.